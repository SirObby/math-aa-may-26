\documentclass{article}
\title{\LaTeX Math notes}
\author{Samuel Hautamäki}
\date{th of October 2024}
\usepackage{mathtools,amssymb,amsthm,gensymb,textcomp}
\begin{document}
  \maketitle
   
  \section{Inverse trigonometric functions}
  p 418\\
  1. $y=\sin(x)$ solve for x || $arcsin() or \sin^{-1}()$\\
  $\sin^{-1}(y)=x$\\
  so $x=arcsin(y)$ is one-to-one function of $\frac{-\pi}{2}\leq x\leq \frac{\pi}{2}$\\
  2) $y=\cos(x) || solve for x || \cos^{-1}() || arccos()$\\
  $\cos^{-1}(y)=x$\\
  so $x=arccos(y)$, one-to-one function in interval $0\leq y\leq\pi$\\
  \\
  For inverse sin and cos\\
  domain $-1\leq x\leq1$\\
  range $0\leq angle\leq\pi$ for arccos\\
  range $-\frac{\pi}{2} angle \leq \frac{\pi}{2}$ for arcsin\\
  3) $y=\tan(x) || solve for x || arctan() || \tan^{-1}()$\\
  $y=tan^{-1}(x)$ is one-to-one function.\\
  domain: all real angles\\
  range: $\frac{-\pi}{2}<yz\frac{\pi}{2}$\\
  Example a) Find $\sin^{-1}(\sin(\frac{\pi}{2}))=\frac{\pi}{2}$\\
  note $arcsin(x)=\sin^{-1}(x)$\\
  b) $\cos^{-1}(\sin(\frac{\pi}{2}))=\cos^{-1}(1)=0$\\
  c) $\tan(\sin^{-1}(-1))=tan(-\frac{\pi}{2})=\frac{\sin(-\frac{\pi}{2})}{-\frac{\pi}{2}}=\frac{-1}{0}$\\
  not defined!\\
  p 420 ex:1-5\\
  \subsection{ex 6p}
  1. a) $\sin^{-1}(\frac{1}{2})=30\degree=\frac{\pi}{6}$\\
  b) $\tan^{-1}(1)=45\degree=\frac{\pi}{4}$\\
  c) $\cos^{-1}(-1)=\pi$\\
  d) $\tan^{-1}(\frac{1}{\sqrt{3}})=30\degree=\frac{\pi}{6}$\\
  e) $\sin^{-1}(0)=0$\\
  f) $\cos^{-1}(\frac{1}{\sqrt{2}})=cos^{-1}(\frac{\sqrt{2}}{2})=\frac{\pi}{4}$\\
  


   
\end{document}
