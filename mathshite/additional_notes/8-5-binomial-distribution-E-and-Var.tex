\documentclass{article}
\title{\LaTeX Math notes}
\author{Samuel Hautamäki}
\date{8th of  May 2025}
\usepackage{mathtools,amssymb,amsthm,gensymb,textcomp}
\usepackage{graphicx}
\graphicspath{ {./images/} }
\begin{document}
  \maketitle
   
  \section{Binomial distribituion E and Var}
  HL p 701\\
  Expected value of binomial $X~B(n,p)$\\
  is $E(X)=n\cdot p$\\
  and Var $Var(X)=n\cdot p\cdot (1-p)=n\cdot p\cdot q$\\
  so deviation $\sigma=\sqrt{n\cdot p\cdot q}$\\
  Example.\\
  Find the mean and deviation of randon variable 'amount of rainy days in a week with p=0.3 success'.\\
  a) mean $E(X)=7\cdot 0.3=2.1$\\
  rainy or not means binomial.\\
  $p=0.3$ and number of trial n=$7$\\
  b) deviation $\sqrt{n\cdot p\cdot q}$\\
  $q=complement=0.7.$\\
  $\sqrt{7\cdot0.3\cdot0.7}=1.212$\\
  hl p binomial E and Var p 717 ex2ab,3cd,5\\
  2. a) $6\cdot 0.4=2.4$\\
  b) $6\cdot0.4\cdot0.6=1.44$\\
  c) the most likely value=find the value with highest probability.\\
  \begin{table}
    \caption{}\label{tab:}
    \begin{center}
      \begin{tabular}[c]{l|l}
        \hline
        \multicolumn{1}{c|}{\textbf{}} & 
        \multicolumn{1}{c}{\textbf{}} \\
        \hline
        0 & $nCr(6,0)\cdot0.4^0\cdot0.6^6$ \\
        1 & $nCr(6,1)\cdot0.4^1\cdot0.6^5$  \\
        2 & $nCr(6,2)\cdot0.4^2\cdot0.6^4$  \\
        3 & $nCr(6,3)\cdot0.4^3\cdot0.6^3$  \\
        4 & $nCr(6,4)\cdot0.4^4\cdot0.6^2$  \\
        5 & $nCr(6,5)\cdot0.4^5\cdot0.6^1$  \\
        6 & $nCr(6,6)\cdot0.4^6\cdot0.6^0$  \\
        \hline
      \end{tabular}
    \end{center}
  \end{table}
  c) highest prob with 3.11 is 2.\\
  3 c) $5\cdot0.5=2.5$\\
  d) $5\cdot0.5\cdot0.5=1.25$\\
  5. a) $12\cdot\frac{1}{3}=4$\\
  b) $1-binomCdf(12,\frac{2}{3},6)=0,82227754350906309449$\\
  c)
  \begin{table}
    \caption{}\label{tab:}
    \begin{center}
      \begin{tabular}[c]{l|l}
        \hline
        \multicolumn{1}{c|}{\textbf{}} & 
        \multicolumn{1}{c}{\textbf{}} \\
        \hline
        1 & $nCr(12,0)\cdot\frac{2}{3}^0\cdot\frac{1}{3}^{12}$\\
        2 & ...  \\
        3 & b \\
        4 & d \\
        5 & b \\
        6 & d \\
        7 & b \\
        8 & d \\
        9 & b \\
        10 & d \\
        11 & b \\
        12 & d \\

        \hline
      \end{tabular}
    \end{center}
  \end{table}
  c)8,p=0.238\\


   
\end{document}
