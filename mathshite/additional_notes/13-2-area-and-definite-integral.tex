\documentclass{article}
\title{\LaTeX Math notes}
\author{Samuel Hautamäki}
\date{th of October 2024}
\usepackage{mathtools,amssymb,amsthm,gensymb,textcomp}
\begin{document}
  \maketitle
   
  \section{Area and definite integral}
  HL p 444\\
  $\int_{a}^{b} f(x) dx is definite intgral$\\
  If f is continouous for $a\leq x \leq b$\\
  and $f(x)>0$ then $\int_{a}^{b} f(x) dx= area$ between function f(x) and x-axis for $a\leq x\leq b$\\
  Properties\\
  $\int_{a}^{b}f(x)dx = \int_{a}^{c}f(x) d(x)+\int_{c}^{b} f(x) dx$\\
  $\int_{a}^{a}=0$\\
  $\int_{a}^{b} f(x)dx=-\int_{b}^{a}$\\
  Inverse if you change the min and max.\\
  SL investigation 5\\
  2a. $\int_{-3}^{2}4dx=20$\\
  b) $\int_{-1}^{2}(-2x+4)dx=9$\\
  SL investigation 6\\
  $\int_{-1}^{3}(x+1) dx=8=A$\\
  $\frac{4\cdot4}{2}=8$\\
  $\int_{-3}^{-1}(x+1)dx=-2=-A$\\
  Integral is area of region.\\
  $\int_{a}^{b}=\begin{cases}
    Area for f(x)>0\\
    -Area for f(x)<0
  \end{cases}$\\
  $\int_{-3}^{3}(x+1)dx=6$\\
  hl p 459 with GDC\\
  1. $\int_{0}^{4}5dx=20$\\
  2. $\int_{0}^{10}(2x-5)dx=50$\\
  3. $\int_{-3}^{3}(3-|x|)dx=9$\\
  4. $\int_{-3}^{4}(\begin{cases}
    4, x\leq 0\\
    4-x,x>0
  \end{cases})=20$\\
  5. 
   
\end{document}
