\documentclass{article}
\title{\LaTeX Math notes}
\author{Samuel Hautamäki}
\date{7th of October 2024}
\usepackage{mathtools,amssymb,amsthm,gensymb,textcomp}
\begin{document}
  \maketitle
   
  \section{Double angle identities}
  hl p405\\
  identity is true for all angles.\\
  $\sin(2\alpha)=2\cdot\sin(\alpha)\cdot\cos(\alpha)$\\
  $\cos(2\alpha)=1-2\cdot\sin^(\alpha)$\\
  $=2\cdot^2(\alpha)-1$\\
  $=\cos^2(\alpha)-\sin^2(\alpha)$\\
  see formulabooklet.\\
  Example. If $\sin(\alpha)=\frac{2}{7}$ and $\frac{\pi}{2}<\alpha<\pi$\\
  a) Find $\sin(2\alpha)$\\
  Lets find $\cos(\alpha)$.\\
  $\sin^2(\alpha)+\cos^2(\alpha)=1$\\
  $\cos(\alpha)=\pm\sqrt{1-\sin^2(\alpha)}$\\
  cos is negative in 2nd quadrant.\\
  $\cos(\alpha)=-\sqrt{1-(\frac{2}{7})^2}=-\sqrt{1-\frac{4}{49}}=-\sqrt{\frac{49-4}{49}}$\\
  $\sqrt{\frac{45}{49}}=\frac{-\sqrt{9\cdot5}}{7}=\frac{-3\sqrt{5}}{7}$\\
  Now, $\sin(2\alpha)=2\cdot\sin(\alpha)\cdot\cos(\alpha)$ subst.\\
  $2\cdot(\frac{2}{7})\cdot(\frac{-3\sqrt{5}}{7})=\frac{-12\sqrt{5}}{49}$\\
  Ans $\sin(2\alpha)=\frac{-12\sqrt{5}}{49}$\\
  $\cos(2\alpha)=1-2\cdot\sin^2(\alpha)$ subst\\
  $1-2\cdot(\frac{2}{7})^2=1-2\cdot\frac{4}{49}$\\
  $=\frac{49-8}{49}=\frac{41}{49}$\\
  c) Find $\tan(2\alpha).$\\
  $\tan(2\alpha)=\frac{\sin(2\alpha)}{\cos(2\alpha)}$\\
  $=\frac{\frac{-12\sqrt{5}}{49}}{\frac{41}{49}}=\frac{-12\sqrt{5}}{49}\cdot(\frac{49}{41})$\\
  $\frac{-12\sqrt{5}}{41}$
  \subsection{HL p 408:ex1...}
  1. a. $\tan(\alpha)+\cot(\alpha)=2\cdot\csc(2\alpha)$\\
  $2\cdot\csc(2\alpha)=2\cdot\frac{1}{2\cdot\sin(\alpha)\cdot\cos(\alpha)}$\\
  $\tan(\alpha)+\frac{1}{\tan(\alpha)}$\\
  $\frac{\sin\alpha}{\cos\alpha}+\frac{1}{\frac{\sin\alpha}{\cos\alpha}}=2\cdot\frac{1}{2\cdot\sin(\alpha)\cdot\cos(\alpha)}$\\
  $\frac{2}{\frac{\sin\alpha}{\cos\alpha}}=2\cdot\frac{1}{2\sin\alpha\cos\alpha}$\\
  $\frac{2}{\frac{\sin\alpha}{\cos\alpha}}\cdot\frac{1}{2}=\frac{1}{2\sin\alpha\cos\alpha}$\\

   
   
\end{document}
