\documentclass{article}
\title{\LaTeX Math notes}
\author{Samuel Hautamäki}
\date{12th of August 2024}
\usepackage{mathtools,amssymb,amsthm,gensymb,textcomp}
\begin{document}
  \maketitle
  
  \subsection{Hmoework}
  hl p85 ex1,c\\
  $y=f(x)=3x^2-12x-5$\\
  concavity up since $a=3>0$\\
  domain = $\mathbb{R}$
  range=$x=\frac{-b}{2a}=\frac{-(-12)}{2\cdot3}=2$\\
  $y=f(2)=3\cdot2^2-12\cdot2-5=-17$\\
  range = $[-17,\infty]$\\
  hl p93 ex2\\
  partial fraction: $\frac{4-x}{x^2+x-2}=\frac{A(x-zeroe)+B(x-zeroe)}{(x-0)\cdot(x-zero)}$\\
  $x^2+x-2=0$\\
  $a=1,b=1,c=-2$\\
  $x=\frac{-b\pm\sqrt{b^2-4a}}{2a}$\\
  $\frac{-1\pm\sqrt{1^2-4\cdot1\cdot(-2)}}{2}$\\
  $x=1 AND x=-2$\\
  $Ans: \frac{4-x}{x^2+x-2}=\frac{A\cdot(x-1)+B(x-(-2))}{(x-1)\cdot(x-(-2))}$\\
  and $4-x=Ax-a+Bx+2B$\\
  $\begin{cases}
    4=-a+2b\\-x=ax+bx ||:x
  \end{cases}$\\
  $\begin{cases}
    4=-a+2b\\
    -1=a+b
  \end{cases}$\\
  $1=b$
  $a+1=-1\\a=-2$\\
  Answer: $\frac{-2\cdot(x-1)+1\cdot(x+2)}{(x-1)\cdot(x+2)}=\frac{-2}{x+2}+\frac{1}{x-1}$ is partial fraction.
  

  \section{Composite function}
  HL Chapter 2.4, p 109\\
  A composite function $f\circ g(x)=f(g(x))$\\
  applies one function (outer f) to the result of another (inner g)\\
  called 'f ball g'\\
  Example\\
  a) $f(x)=\sqrt{x-2}+3$ and $and g(x)=\frac{4x}{7}$\\
  $f\circ g(x)=f(g(x))=f(\frac{4x}{7})$\\
  $\sqrt{\frac{4x}{7}-2}+3$\\
  b) but $g\circ f(x)=g(f(x))=g(\sqrt{x-2}+3)=\frac{4\cdot(\sqrt{x-2}+3)}{7}$\\
  Note:  $f\circ g\neq g\circ f$\\
  
  \section{hl p111}
  1. a) $f(x)=3x; g(x)=\sqrt{x}$\\
  i) $g(f(1))=\sqrt{3\cdot1}=\sqrt{3}$\\
  ii) $f(g(2))=3\cdot\sqrt{2}$\\
  iii) $f(g(x))=3\cdot\sqrt{x}$\\
  iv) $g(f(x))=\sqrt{3x}$\\
  b. $f(x)=5-3x; g(x)=x^2+4$\\
  i) $g(f(1))=(5-3\cdot1)^2+4=8$\\
  ii) $f(g(2))=5-3\cdot(8)=-19$\\
  iii) $f(g(x))=5-3*(x^2+4)$\\
  iv) $g(f(x))=(5-3x)^2+4$\\
  c. $f(x)=x+1;g(x)=\sqrt{2x-1}$\\
  2. i) a) $f(x)=x^2+x;g(x)=2-3x$\\
  domain = $\mathbb{R}$\\
  range = 
  
  

   
\end{document}
