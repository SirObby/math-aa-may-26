\documentclass{article}
\title{\LaTeX Math notes}
\author{Samuel Hautamäki}
\date{th of October 2024}
\usepackage{mathtools,amssymb,amsthm,gensymb,textcomp}
\begin{document}
  \maketitle
  
  \section{Trigonometric equations}
  hl p391 ex4\\
  Cos(x)=2.15\\
  $x=cos^{-1}(2.15)=undef$ \\
  no solution since cos means x-value in unit circle and $-1 \leq x \leq 1$\\
  ex7. $4\cdot sin^2(x)=1 ||:4$ \\
  $sin^2(x)=\frac{1}{4} ||\pm sqrt$\\
  $sin(x)=\frac{\pm1}{2}$ \\
  $sin(x)=\frac{1}{2}$ or $sin(x)=\frac{-1}{2} and 0<x<2\pi$\\
  $x=sun^{-1}(\frac{1}{2})=30$\\
  $x= \begin{cases}
  \frac{\pi}{6}+h\cdot 2 \pi \\
  \pi -(\pi /6)+n\cdot 2 \pi=5 \pi /6 +n\cdot 2\pi  
  \end{cases}$ or x=$x=\pi-(-\pi/6)=\frac{7\pi}{6}$\\
  ans$x=\frac{\pi}{6},x=\frac{5 \pi }{6}, x=\frac{7\pi}{6} and x=\frac{-\pi}{6}+2\pi=\frac{-\pi}{6}+\frac{12\pi}{6}=\frac{11\pi}{6}$

  \subsection{2.2 handout}
  a) $\cos 120\degree =-0.5$\\
  $\sin 120\degree=0.8$\\
  b) $\cos -30\degree=0.9$\\
  $\sin -30=-0.5$\\
  c) $\cos 250\degree=-0.3$\\
  $\sin 250\degree=-0.9$\\
  \section{HL Repiprocal trigonometric}
  p 400
  $\frac{1}{\sin x}=cosec (x)=\frac{hypotenuse}{opposite}$\\
  $\frac{1}{\cos x}=sec (x)=\frac{hypotenuse}{adjacent}$\\
  $\frac{1}{tan x}=cot (x)=\frac{adjacent}{opposite}$\\
  Example a) simplify $sec(\frac{\pi}{3})-cot(\frac{\pi}{6}).$\\
  paper 1: $\frac{1}{cos(\frac{\pi}{3})}-\frac{1}{tan \frac{\pi}{6}}=\frac{1}{\frac{1}{2}}-\frac{1}{\frac{1}{\sqrt{3}}}=\frac{2}{1}-\frac{\sqrt{3}}{1}=2-\sqrt{3}$\\
  paper 2: $sec(\frac{\pi}{3})-cot(\frac{\pi}{6})=0.267$\\
  b) calculate\\
  $cosec(-\frac{\pi}{4})+3\cdot sec(-\frac{\pi}{6})$\\
  $=\frac{1}{sin- \frac{\pi}{4}}+3\cdot(\frac{1}{cos(-\frac{\pi}{6})})$\\
  $=-(\frac{1}{\frac{1}{\sqrt{2}}})+3\cdot(\frac{1}{-(\frac{\sqrt{3}}{2})})$ use unit circle with $\pm$\\
  $=\sqrt{2}+3\cdot\frac{2}{\sqrt{3}}=$ no roots in divisor\\
  $-\sqrt{2}+3\cdot\frac{2\cdot\sqrt{3}}{\sqrt{3}^2}=-\sqrt{2}+2\cdot\sqrt{3}$\\
  $2\sqrt{3}-\sqrt{2}$\\
  $csc(\frac{-\pi}{4})+3\cdot sec(\frac{-\pi}{6})=2.04989$
  \subsection{excercises p 400: 1-3}
  1. $\frac{1}{\cos \frac{\pi}{6}}-\frac{1}{\tan \frac{\pi}{3}}$\\
  $\frac{1}{\frac{\sqrt{3}}{2}}-\frac{1}{\tan \frac{1}{2}}$\\
  $\frac{2}{\sqrt{3}}-\frac{2}{1}$\\
  $\frac{2}{\sqrt{3}}-2$\\
  Ans: 0.577\\
  2. $csc (\frac{-2\pi}{3})+2\cdot\tan \frac{7\pi}{6}$\\
  $\frac{1}{-2\cdot\frac{\sqrt{3}}{2}}+2\cdot7\cdot\frac{1}{\sqrt{3}}$\\
  $\frac{-2\cdot3}{2}+14\cdot\frac{1}{3}$\\
  $1-\frac{2}{\sqrt{3}}$\\
  3a) $\cos x\cdot\tan x$\\
  $\cos x\cdot(\frac{\sin x}{\cos x})=\sin x$\\
  3b) $cot x\cdot sec x=\frac{1}{\tan x}\cdot\frac{1}{\cos x}=\frac{\cos x}{\sin x}\cdot(\frac{1}{\cos x})$\\
  $=\frac{1}{\sin x}$\\
  4. $cosec (x)=\frac{13}{5} draw triangle \frac{hypotenuse}{opposite}$\\

  
  

  
  

   
\end{document}
