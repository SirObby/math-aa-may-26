\documentclass{article}
\title{\LaTeX Math notes}
\author{Samuel Hautamäki}
\date{th of September 2024}
\usepackage{mathtools,amssymb,amsthm,gensymb,textcomp}
\begin{document}
  \maketitle
   
  \section{Revision}
  1) Function is relation that connects every x from domain into exactly one y from range.\\
  (vertical line intersects only once with function, If vertical line has more intersection points, then relation is not a function.)\\
  2) Composite function $f\circ g(x)=f(g(x))$\\
  inside brackets there is inner functio and outside the brackets there is outer function.\\
  3) inverse function\\
  $y=f(x)$ solve for x\\
  $x=f^{-1}(y)$ interchange x and y\\
  $Ans: y=f^{-1}(x)$\\
  definition of inverse: $f(f^{-1}(x))=x$\\
  4) limit $\lim_{x\to a}(f(x))=b$\\
  exists if b is a number, not 2 numbers, not $\infty$\\
  limit is found from table, graph or algebric way\\
  $\lim_{x\to1}(\frac{2x-2}{x^2-1})=to to substitute\frac{}{1^2-1}=\frac{}{0}$= try to cancel factors\\
  $=\lim_{x\to1}(\frac{x\cdot(x-1)}{(x-1)\cdot(x+1)})=\frac{2}{1+1}$=1\\
  HL\\
  continuous function has $\lim_{x\to a^-}(f(x))=\lim_{x\to a^+}(f(x))=f(a)$\\
  5) differential calculus\\
  deriverative gives slope=gradient of tangent\\
  $=\frac{d}{dx}(y)=y'(x)=f'(x)$\\
  HL first principle method:\\
  $\lim_{h\to0}(\frac{f(a+h)-f(a)}{h})=f'(a)$\\
  SL rules:\\
  power rule: $D x^n=n\cdot x^{n-1}$\\
  $D \frac{3}{\sqrt{x}}=D 3\cdot x^{-\frac{1}{2}}$\\
  $3\cdot(\frac{-1}{2})\cdot x^{\frac{-3}{2}}=\frac{-3}{2\cdot\sqrt{x^3}}$\\
  Constant: $D 3 =0$\\
  but factor $D 3\cdot x=3$\\
  $D 3\cdot f(x)=3\cdot f'(x)$\\
  product rule: $D (f\cdot g)=f'(x)\cdot g(x)+f(x)\cdot g'(x)$\\
  quotiont rule: $D (\frac{f(x)}{g(x)})=\frac{f'(x)\cdot g(x)-f(x)\cdot g'(x)}{g(x)^2}$\\
  chain rule for brackets!\\
  $D f(g(x))=f'(g(x))\cdot g'(x)$\\
  tangent line slope $m=f'(x_0)$\\
  $y-y_0=m\cdot(x-x_0)$\\
  normal is perpendicular to tangent line\\
  normal has slope m=$\frac{-1}{f'(x_0)}=\frac{-1}{slope of tangent}$\\
  function increases $f'(x)>0$\\
  decreases $f'(x)<0$\\
  at minimum $f'(x)=0$ and $f''(X)>0$\\
  at maximum $f'(x)=0$ and $f''(x)<0$\\
  at point where $f''(x)=0$ there is inflection\\


   
\end{document}
