\documentclass{article}
\title{\LaTeX Math notes}
\author{Samuel Hautamäki}
\date{th of April 2025}
\usepackage{mathtools,amssymb,amsthm,gensymb,textcomp}
\usepackage{graphicx}
\graphicspath{ {./images/} }
\begin{document}
  \maketitle
   
  \section{Discrete random variable}
  HL p 696\\
  A random variable = quantity whose value depends on chance (X,Y,Z)\\
  distribution assigns probability to each of these values.\\
  Example Toss 3 coins. Discrete random variable X is the number of tails obtained.\\
  X= number of tails:\\
  distribution, Table one.\\
  \begin{table}
    \caption{}\label{tab:}
      \begin{tabular}[c]{l|l}
        \hline
        \multicolumn{1}{c|}{\textbf{}} & 
        \multicolumn{1}{c}{\textbf{}} \\
        \hline
        X & P(X=x) \\
        0 & P(Head,Head,Head)=$\frac{1}{2}\cdot\frac{1}{2}\cdot\frac{1}{2}=\frac{1}{8}$ \\
        1 & (THH,HTH,HHT)=$\frac{1}{2}\cdot\frac{1}{2}\cdot\frac{1}{2}+\frac{1}{2}\cdot\frac{1}{2}\frac{1}{2}+\frac{1}{2}\cdot\frac{1}{2}\cdot\frac{1}{2}=\frac{3}{8}$ \\
        2 & (TTH,HTT,HTH)=$\frac{1}{2}\cdot\frac{1}{2}\cdot\frac{1}{2}+\frac{1}{2}\cdot\frac{1}{2}\frac{1}{2}+\frac{1}{2}\cdot\frac{1}{2}\cdot\frac{1}{2}=\frac{3}{8}$\\
        3 & (TTT)=$\frac{1}{2}\cdot\frac{1}{2}\cdot\frac{1}{2}=\frac{1}{8}$ \\
        \hline
      \end{tabular}
  \end{table}
  sum of all P=$\frac{8}{8}=1=100\%$\\
  table has Probability Distribution Function=PDF\\
  calculator also has CDF=Cumulative Distribution Function.\\
  mean=$\mu=$expected value=E(X)=$\sum x_i\cdot p_i$\\
  Where i refers to row of distribution table.\\
  Example Fidn the expected value fo tossing 3 coins.\\
  $E(X)=0\cdot (\frac{1}{8}+1\cdot\frac{3}{8}+2\cdot\frac{3}{8}+3\cdot\frac{1}{8}$\\
  $=\frac{0\cdot (\frac{1}{8}+1\cdot\frac{3}{8}+2\cdot\frac{3}{8}+3\cdot\frac{1}{8}}{8}=\frac{3}{2}$\\
  HL only variance $\sigma^2=Var(X)=E(X^2)-(E(X))^2$ or\\
  $\sum x^2\cdot p-\mu ^2$ (see infobooklet)\\
  standard deviation=$\sigma=\sqrt{Var(X)}$\\
  mode=the most typical value, with highest $p_i$.\\
  Example 3. Find mean, mode, and standard deviation of tossing 3 coins.\\
  $mean=expected value=example 2=\frac{3}{2}=1.5$ coins.\\
  mode=$most typical=1 tail and 2 tails are both modes!$\\
  standard deviation=$\sqrt{Var(X)}$\\
  $Var(X)=\sum(x_i-\mu)^2\cdot P(X_i)$\\
  $(0-3/2)^2\cdot\frac{1}{8}+(1-\frac{3}{2})^2\cdot\frac{3}{8}+(2-\frac{3}{2})^2\cdot\frac{3}{8}+(3-\frac{3}{2})^2\cdot\frac{1}{8}$\\
  $=0.75$\\
  $deviation=\mu=\sqrt{0.75}=0.86$\\
  $\sum(X=x)=1$\\
  HL p 699 ex 1,2 p 701 ex 1,2 p 705 ex 1,2\\
  1. a) $0+0.1+0.2+0.3+0.4=1$ yes, valid.\\
  b) $0.3+0.3+0.3+0.3+0.3=1.5$ not valid.\\
  2. i. (Table 2), invalid. \\
  \begin{table}
    \caption{}\label{tab:}
    \begin{center}
      \begin{tabular}[c]{l|l}
        \hline
        \multicolumn{1}{c|}{\textbf{}} & 
        \multicolumn{1}{c}{\textbf{}} \\
        \hline
        0 & $\frac{1}{2}0^3=0$ \\
        1 & $\frac{1}{2}1^3=0.5$ \\
        2 & $\frac{1}{2}2^3=4$\\
        \hline
      \end{tabular}
    \end{center}
  \end{table}
  ii. (Table 3) Invalid.\\
  \begin{table}
    \caption{}\label{tab:}
    \begin{center}
      \begin{tabular}[c]{l|l}
        \hline
        \multicolumn{1}{c|}{\textbf{}} & 
        \multicolumn{1}{c}{\textbf{}} \\
        \hline
        0 & $\frac{0!}{5\cdot0+2}=0.5$ \\
        1 & $\frac{1!}{5+2}=0,14285714285714285714$ \\
        2 & $\frac{2!}{5\cdot2+2}=0,16666666666666666667$\\
        3 & $\frac{3!}{5\cdot3+2}=0,35294117647058823529$\\
        4 & $\frac{4!}{5\cdot4+2}=1,09090909090909090909$\\
        \hline
      \end{tabular}
    \end{center}
  \end{table}
  701,1. \\
  $E(X)=\sum (x_i\cdot p_i)$\\
  $0\cdot\frac{1}{6}+1\cdot\frac{1}{6}+2\cdot\frac{1}{3}+3\cdot\frac{1}{4}+4\cdot\frac{1}{12}=1,91666666666666666667$\\
  2. $0.1+0.2+k+2k+k-0.1=1$\\
  k≈0.21951\\
  $E(X)=\sum (x_1\cdot p_1)$\\
  $4\cdot0.1+5\cdot0.2+6\cdot k+ 7\cdot 2k+8\cdot (k-0.1)$\\
  $=55,94628$\\
  705,1.\\
  a) $1\cdot\frac{1}{8}+2\cdot\frac{3}{8}+3\cdot\frac{1}{2}$\\
  $=2,375$\\
  b) $E(5X)$\\

  

   
\end{document}
