\documentclass{article}
\title{\LaTeX Math notes}
\author{Samuel Hautamäki}
\date{19th of August 2024}
\usepackage{mathtools,amssymb,amsthm,gensymb,textcomp}
\begin{document}
  \maketitle
  
  \subsection{Homework}
  HL p 223: ex 5,6,7\\
  $f(x)=\begin{cases}
    x-3,x<2\\
    x+1,x>2
  \end{cases}$\\
  $\lim_{x\to 2}(x-3)2-3=-1$\\
  $\lim_{x\to 2}(x+1)=2+1=3$
  $\lim_{x\to 2}(f(x))= is not a unique (-1\neq3) so limit does not exist.$

  \section{Continuity}
  (=funktion jatkuvuus)
  HL page 224\\
  A continuous function $f(x)$ has\\
  $\lim_{x\to c^-}(f(x))=\lim_{x\to c^+}=f(c)$\\
  so function has limit from left and right and it shte same than function value at that point.\\
  Example\\
  Draw function that is continous at x=c, but non-continuous at x=a and x=b\\
  Example 2. is $g(x)=\begin{cases}
    x^2,x>0\\
    x-2,x<0\\
    3, x=0
  \end{cases}$ a continous function?
  Ans: Not a continous function\\
  $\lim_{x\to 0^-}(f(x))\neq\lim_{x\to 0^+}(f(x))\neq f(0)$\\
  \section{p 227 ex1,2,5,6,8a}
  1. f(x)=$\begin{cases}
    2x+1,x<3\\
    3x-2,x\geq2
  \end{cases}$\\
  left limit = $\lim_{x\to3^-}(2x+1)=2\cdot3+1=7$\\
  right limit = $\lim_{x\to3^+}(3x-2)=3\cdot3-2=9-2=7$\\
  $f(3)=3\cdot3-2=9-2=7$\\
  Ans f is continous\\
  2. $f(x)=\begin{cases}
    x^2-1,x\leq2\\
    2x-1,x>2
  \end{cases}$\\
  $\lim_{x\to2^-}(x^2-1)=4-1=3$\\
  $\lim_{x\to2^+}(2x-1)=4-1=3$\\
  $f(2)=2^2-1=3$\\ Ans is continuous.
  f is continous\\
  5. $f(x)=\begin{cases}
    \frac{x^2-1}{x^2+x-2}, x\neq1,x\neq-2\\
    \frac{2}{3}, x=1\\
    4, x=-2
  \end{cases}$\\

  
\end{document}
