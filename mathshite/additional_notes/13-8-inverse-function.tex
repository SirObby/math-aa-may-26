\documentclass{article}
\title{\LaTeX Math notes}
\author{Samuel Hautamäki}
\date{13th of August 2024}
\usepackage{mathtools,amssymb,amsthm,gensymb,textcomp}
\begin{document}
  \maketitle
   
  \subsection{homework}
  (sl book)p90 ex3\\
  a) $g(f(0))=g(0)=-4$\\
  b) $g(f(1))=g(1)=-3$\\
  (hl)\\
  hl p 111 ex 4\\
  $f(x)=3x+a;g(x)=\frac{x-4}{3}$\\
  equation $f(g(x)) = g(f(x))$ subst\\
  $f(\frac{x-4}{3})=g(3x+a)$\\
  $3\cdot(\frac{x-4}{3})+a=\frac{(3x+a)-4}{3}$\\
  $x-4+a=\frac{3x+a-4}{3} || \cdot3$\\
  $3x-12+4a=3x+a-4$\\
  $2a=8$\\
  $a=4$\\
  ex 4a) \\
  $f(g(x))=x^2-2x+3$ \\
  $f(x)=x+3$\\
  $g(x)=x^2-2$\\

  \section{Inverse function}
  HL p112\\
  Inverse function $f^{-1}(x)$ reverses the action of $f(x)$\\
  Definition: $f^{-1}(f(x))=x=f(f^{-1}(x))$.\\ 
  Note! f must be one-to-one function so that $f^{-1}$ exists.\\
  Example find the inverse of $f(x)=\sqrt{x-2}+3$\\
  1) Write $f(x)=y$\\
  2) solve for x.\\
  3) interchange x and y.\\
  4) write $f^{-1}(x)=y=...$\\
  Ans: \\
  $y=\sqrt{x-2}+3 ||-3$\\
  $y-3=\sqrt{x-2} || ()^2$\\
  $(y-3)^2=x-2$\\
  $x-2=(y-3)^2$\\
  $x=(y-3)^2+2$ change x and y\\
  $y=(x-3)^2+2$ for $f^{-1}(x)=(x-3)^2+2$\\
  or $f^{-1}(y)=(y-3)^2+2, x\geq3$
  b) Sketch $f(x)$ and $f^{-1}(x)$\\
  function and the inverse function are symmetrical about line $y=x$.\\
  Example 2. Determine whether f and g are inverse. $f(x)=\sqrt{x^2-7}+x, g(x)=2-x$\\
  Lets find\\
  $f(g(x))=f(2-x)=\sqrt{(2-x)^2-7}+(2-x)\neq x$\\
  since $\sqrt{(2-x)^2-7}\neq0 for all x.$\\
  Ans: No.\\
  
  \section{hl book p 116 ex a only }
  1. a) $f(x)=\{(4,2)(0,2)(-2,2),(2,2)\}$\\
  does not have inverse.\\
  2. a) $y=5x-1$\\
  $y=5x-1 || +1$\\
  $x+1=5y$\\
  $5y=x+1$\\
  $y=\frac{x+1}{5}$.\\
  $f^{-1}(x)=\frac{x+1}{5}$\\
  $f^{-1}(f(x))=\frac{(5x-1)+1}{5}$\\
  3. a) $f(x)=y=(x-2)^2$\\
  1) Write $y=(x-2)^2 || \sqrt{}$ solve for x\\
  $\pm\sqrt{y}=x-2$\\
  $2\pm\sqrt{y}=x$ interchange\\
  $y=2\pm\sqrt{x}$\\
  inverse function is\\
  1) $f^{-1}(x)=2+\sqrt{x}, for x$\\
  or 2) $f^{-1}(x)=2-\sqrt{x}, for x$\\
  domain of f = range of $f^{-1}=[2,\infty]$ in case 1\\
  or $[-\infty, 2]$
  
  
   
\end{document}
