\documentclass{article}
\title{\LaTeX Math notes}
\author{Samuel Hautamäki}
\date{th of October 2024}
\usepackage{mathtools,amssymb,amsthm,gensymb,textcomp}
\begin{document}
  \maketitle
   
  \section{Integration by parts}
  HL p 504\\
  revise: $D(f(x)\cdot g(x))=f'(x)\cdot g(x)+g'(x)\cdot f(x)$\\
  so $\int(f\cdot g')dx=f\cdot g-\int(f'\cdot g)dx$\\
  that is in formula booklet in the form\\
  5.16 $\int (u\cdot(\frac{dv}{dx}))dx=u\cdot v-\int(v\cdot\frac{du}{dx})dx$\\
  Example. Find $\int (3x\cdot e^x)dx=?$\\
  substitute $u=3x$ and $v'=e^x$\\
  so $u'=3$ and $v=e^x+C$\\
  partial integration gives\\
  $\int(3x\cdot e^x)dx=3x\cdot (e^x+C)-\int(e^x\cdot3)dx$\\
  $=3xe^x-3\cdot e^x+C$\\
  Example 2.\\
  $\int(sin(x)(2-5x))dx=?$\\
  partial integration $\int(u\cdot v')dx=u\cdot v-\int(u'\cdot v)dx$\\
  subt. $u=2-5x$ and $v'=sin(x)$\\
  so $u'=-5$ and $v=-cos(x)$\\
  now $\int(sin(x)\cdot(2-5x))dx=(2-5x)\cdot(-cos))-\int(-5\cdot(-cos)) dx$\\
  $=-2cos(x)+5x\cdot cos(x)-5 sin(x)+C$\\
  ex 7 p 507\\
  ex 1. $\int (2x\cdot e^x)dx$\\ 
  partial integration formulabooklet\\
  $\int(u\cdot v')dx=u\cdot v-\int(u'\cdot v)dx$\\
  subst. $u=2x$ $v'=e^x$\\
  $u'=2 v=e^x$\\
  $\int(2x\cdot e^x)dx=(2x)\cdot e^x-\int(2\cdot e^x)dx$\\
  $2xe^x-2e^x+C$\\
  ex 2. $\int (3x sin(x))dx$\\
  $u=3x$ $v'=sin(x)$\\
  $u'=3 $ $v=-cos(x)$\\
  $\int(3x\cdot sin(x))dx=3x\cdot -cos-\int(3\cdot -cos(x))dx$\\
  $-3xcos(x)+3x sin(x)+C$\\
  ex 3. $\int(1-2x)e^xdx$\\
  $u=1-2x$ $v'=e^x$\\
  $u'=-2$ $v=e^x$\\
  $\int((1-2x)\cdot)dx=(1-2x)\cdot e^x-\int(-2\cdot e^x)dx$\\
  $=(1-2x)e^x+2e^x+C$\\
  ex4. $\int((2-x)\cdot sin(2-x))dx$\\
  $u=2-x$ $v'=sin(2-x)$\\
  $u'=0-1$\\
  $v=\int sin(2-x)dx=-1\cdot\int(-1\cdot sin(2-x))dx$\\
  $=-1\cdot (-cos(2-x))=cos(2-x)$\\
  now $\int((2-x)\cdot sin(2-x))dx=(2-x)\cdot cos(2-x)-\int(-1\cdot cos(2-x))dx$\\
  $=(2-x)cos(2-x)-sin(2-x)+C$\\
  ex 5. $\int (\frac{1+2x}{3}\cdot (sec^2(\frac{1}{2}\cdot x)))dx=?$\\
  $u=\frac{1+2x}{3})=\frac{1}{3}+\frac{2}{3}\cdot x$\\
  $u'=0+\frac{2}{3}$\\
  $v'=sec^2(\frac{1}{2}x)$\\
  $v=\int(sec^2(\frac{1}{2}x))dx=2\cdot \int (\frac{1}{2}\cdot sec^2(\frac{1}{2}x))$\\
  $2\cdot tan(\frac{1}{2}x)$\\
  now subst for partial integration.\\
  $\int (\frac{1+2x}{3}sec^2(\frac{x}{2}))dx$\\
  $=(\frac{1+2x}{3})\cdot 2tan(\frac{x}{2})-\int(\frac{2}{3}\cdot (2\cdot tan(\frac{1}{2}x)))dx$\\
  
  
   
\end{document}
