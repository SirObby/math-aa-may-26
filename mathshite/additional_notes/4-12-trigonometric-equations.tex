\documentclass{article}
\title{\LaTeX Math notes}
\author{Samuel Hautamäki}
\date{th of October 2024}
\usepackage{mathtools,amssymb,amsthm,gensymb,textcomp}
\begin{document}
  \maketitle
   
  \section{Trigonometric equations}
  HL p420\\
  identities:$\tan(x)=\frac{sin(x)}{cos(x)}$\\
  $\sin^2x+\cos^2x$\\
  paper 2: $nsolve(f(x)=0,x,guess)$\\
  OR graph with intersection points....\\
  paper 1:\\
  Example solve $4\cos(x)-3sec(x)=2\tan(x), -180\degree\leq x\leq180\degree$\\
  simplify so that equation has only 1 type of trigonometric function\\
  $4\cos-3\cdot\frac{1}{\cos(x)}=2\cdot\frac{\sin(x)}{\cos(x)} || \cdot\cos(x)\neq0$\\
  $4\cdot(\cos(x))^2-3=2\cdot\sin(x)$\\
  and $\cos^2(x)=-1\sin^2x$\\
  $4\cdot(1-sin^2x)-3=2\cdot\sin x$\\
  $4-4\sin^2x-3-2\cdot\sin x=0$\\
  $-4\cdot\sin^2x-2\cdot\sin x+1=0$\\
  $a=-4,b=-2,c=1, guadratic formula$\\
  $\sin(x)=\frac{-(-2)\pm\sqrt{(-2)^2-4\cdot-4\cdot1}}{2\cdot(-4)}$\\
  $x=\sin^{-1}(\frac{2+\sqrt{20}}{-8})=-54$\\
  or $180--54=234$\\
  $x=\sin^{-1}(\frac{2-\sqrt{20}}{-8}=18$\\
  or $180-18=162$\\
  Ans: $x is -54\degree,18\degree,162\degree,234\degree$\\
  Paper 2.\\
  $nSolve(4\cos(x)-3sec(x)=2\tan(x),x)=18\degree$\\
  Or intersection with x-axis/zeroes\\
  p 421, ex 1,a,b,c,d paper 1 + 2\\
  1a) $cosec^2x=3 cot x-1$\\
  identities: $1+\cot^2x=cosec^2x$\\
  $1+cot^2x-3\cdot cot+1=0$\\
  $cot^2x-3\cdot cot+2=0$ guadratic formula\\
  $a=1,b=-3,c=2$\\
  $\cot x=\frac{-(-3)\pm\sqrt{(-3)^2-4\cdot1\cdot2}}{2\cdot1}$\\
  $\cot x=\frac{3\pm1}{2}$\\
  $cot x=2 or cot x=1$\\
  $\frac{1}{tan x}=2$ so $tan x=1/2 x=26.56$\\
  $\frac{1}{tan x} so tan x=1 x=45$\\
  paper 2: graph\\
  Ans: 26.6,45\\
  b) $2\tan\theta=3+5\cot\theta$\\
  Ans: $-45\degree+n\cdot180\degree or x=68.2\degree+n\cdot180\degree$\\
  c) 
  % $\frac{1}{\sin^2(x)}=3\cdot\frac{\sin(x)}{\cos(x)}-1$\\

  

   
\end{document}
