\documentclass{article}
\title{\LaTeX Math notes}
\author{Samuel Hautamäki}
\date{23rd of August 2024}
\usepackage{mathtools,amssymb,amsthm,gensymb,textcomp}
\begin{document}
  \maketitle
   
  \section{Derivative from the first principles}
  $f'(a) =$ derivative is point $x=a$, so slope of tangent\\
  $=\frac{change in y}{change in x}$ on tangent line\\
  $=\lim_{h\to 0}(\frac{f(a+h)-f(a)}{h})$\\
  see formula booklet\\
  Example. Find the derivative of $f(x)=x^2$ at point $x=3$\\
  $f'(3)=\lim_{h\to 0}(\frac{f(3+h)-f(3)}{h})$ subst\\
  $=\lim_{h\to0}(\frac{(3+h)^2-3^2}{h})$ simplify\\
  $\lim_{h\to0}(\frac{3^2+2\cdot3\cdot h+h^2-3^2}{h})$\\
  note binomial formula $(a+b)^2=a^2+2ab+b^2$\\
  $=\lim_{h\to0}(\frac{6h+h^2}{h})=\lim_{h\to0}(\frac{(6+h)\cdot h}{h})$\\
  cancell only from product form.\\
  $=\lim_{h\to0}(6+h)=6+0=6$\\
  OR calculator graph slope of tangent\\
  Note! Given a function is differentiable then it is continuous.\\
  If function is non-continuous then f is not differentiable\\
  And i f is continuous there might be a sharp corner without unique tangent, so no differentiable.\\
  p 242 ex 13.\\
  \section{p 243 }
  ex 1a) $f(x)=x^2+x-2$\\
  $f'(0)=\lim_{h\to0}(\frac{f(0+h)-f(0)}{h})$ subst.\\
  $=\lim_{}(\frac{h^2+h-2 -(0^2+0-2)}{h})$\\
  $=\lim_{h\to0}(\frac{h^2+h-2+2}{h})$\\
  $=\lim_{h\to0}(\frac{h\cdot(h+1)}{h})=\lim_{h\to0}(h+1)=1$\\
  ex p. 243-244
  1b) $f(x)=2-x+3x^2$\\
  $f'(-1)=\lim_{h\to0}(\frac{f(-1+h)-f(-1)}{h})$\\
  $\lim_{h\to0}(\frac{2-(-1+h)+3(-1+h)^2-(2-(-1)+3\cdot(-1)^2)}{h})$\\
  $=\lim_{h\to0}(\frac{2+1-h+3(1-2h+h^2)-2-1+3}{h})$\\
  $=\lim_{h\to0}(\frac{-h+3-6h+3h^2-3}{h})=\lim_{h\to0}(\frac{h\cdot(-1-6+3h)}{h})$\\
  $=\lim_{h\to0}(-7+3h)$=-7\\
  c) $f(x)=\frac{-2}{x}$\\
  $f'(1)=\lim_{h\to0}(\frac{f(1+h)-f(1)}{h})$\\
  $=\lim_{h\to0}(\frac{\frac{-2}{1+h}-\frac{-2}{1}}{h})$\\
  $=\lim_{h\to0}(\frac{\frac{-2}{1+h}+\frac{2\cdot(1+h)}{1+h}}{h})$\\
  $=\lim_{h\to0}(\frac{-2+2+2h}{1+h}\cdot(\frac{1}{h}))$\\
  $=\lim_{h\to0}(\frac{2}{1+h})=\frac{2}{1+0}=2$\\
  1d) $f(x)=\sqrt{x+1}$\\
  $f'(3)=\lim_{h\to0}(\frac{f(3+h)-f(h)}{h})$\\
  $=\lim_{h\to0}(\frac{\sqrt{3+h+1}-\sqrt{3+1}}{h})$\\
  $=\lim_{h\to0}(\frac{\sqrt{4+h}-\sqrt{4}}{h}\cdot\frac{\sqrt{4+h}+\sqrt{4}}{\sqrt{4+h}+\sqrt{h}})$\\
  $=\lim_{h\to0}(\frac{(\sqrt{4+h}^2)-(\sqrt{4})^2}{h\cdot(\sqrt{4+h}+\sqrt{4})})$\\
  $=\lim_{h\to0}(\frac{4+h-4}{h(\sqrt{4+h}+\sqrt{4})})$\\
  $=\lim_{h\to0}(\frac{1}{\sqrt{4+h}+\sqrt{5}})=\frac{1}{\sqrt{4+0}+\sqrt{5}}=\frac{1}{2+2}=\frac{1}{4}$\\
  ex 2.\\
  a) $average velocity=\frac{change in displacement}{change in time}$\\
  $=?=\frac{f(a+h)-f(a)}{(a+h)-a}$\\
  $f(t)=8+2t^2$ where t=time from t=a into t=a+h\\
  $=\frac{(8+2(a+h)^2)-(8-2a^2)}{h}$\\
  $\frac{8+2(a^2+2ah+h^2-8+2a^2)}{h}$\\
  $\frac{8+2a^2+4ah+2h^2-8-2a^2}{h}$\\
  $4a+2h$


\end{document}
