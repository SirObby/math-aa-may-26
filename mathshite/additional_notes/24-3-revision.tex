\documentclass{article}
\title{\LaTeX Math notes}
\author{Samuel Hautamäki}
\date{th of October 2024}
\usepackage{mathtools,amssymb,amsthm,gensymb,textcomp}
\begin{document}
  \maketitle
   
  \section{Revision of 4th math topics}
  are between f(x) and x-axis=$\int_{a}^{b}f(x)dx$ where $a<x<b$.\\
  integral $F(x)=\int f(x)dx + C$\\
  where $F'(x)=f(x)$ so antidifferentiation!\\
  rules:\\
  formula booklet\\
  note product\\
  $\int cos(2x)\cdot sin(2x)^1dx$\\
  inner function $D sin(2x)=cos(2x\cdot D((2x))$\\
  $=cos(2x)\cdot 2$\\
  $=\frac{2}{2}\cdot\int (cos(2x)\cdot sin(2x)^1)dx$\\
  $\frac{1}{2}\int (2\cdot cos(2x)\cdot sin(2x)^1)dx$\\
  $=\frac{1}{2}\cdot\frac{(sin (2x)^2)}{2}+C=\frac{1}{4}sin^22x+C$\\
  power rule:\\
  $\int x^n dx=\frac{x^{n+1}}{n+1}+C$\\
  and $\sqrt(x)=x^{1/2} or \frac{1}{\sqrt[3]{x}}=x^{-1/3}$\\
  $\int \frac{1}{x}dx=\int x^{-1}dx=ln|x|+C$\\
  $\int(e^{f(x)}\cdot f'(x))dx=e^{f(x)}+C$\\
  HL: partial integration p substitutions u=...\\
  see formula booklet\\

  volume=$\int_{a}^{b}(\pi \cdot(f(x))^2)dx, a<x<b$\\
  or about y-axis V$=\pi\int_{y1}^{y2}(g(y))^2dy$\\
  where $x=g(y)$ and $y=f(x)$\\

  revision:\\
  1. area between $y=2+x-x^2$ and $y=2-3x+x^2$\\
  intersection $y=y$\\
  $2+x-x^2=2-3x+x^2$\\
  $4x-2x^2=0$\\
  $x\cdot(4-2x)=0$\\
  $x=0$ or $42x=0$, so $4=2x$, x=2\\
  Area$\int_{0}^{2}(upper-lower)dx=\int_{0}^{2}(2+x-x^2-(23x+x^2))dx$\\
  $\int_{0}^{2}(4x-2x^2)dx=[\frac{4x^2}{2}-\frac{2x^3}{3}]$\\
  $=\frac{4\cdot 2^2}{2}-\frac{2\cdot2^3}{3}-(\frac{4\cdot0^2}{2}-\frac{2\cdot0^3}{3})$\\
  $=8-\frac{16}{3}=\frac{8}{3}$\\
  2. partial integration.\\
  $\int_{1}^{e}x^2 ln x dx$\\
  $u=ln(x)$ $u'=1/x$\\
  $v'=x^2$ $v=x^3/3$\\
  $ln(x)\cdot(\frac{x^3}{3})-\int_{1}^{e}(\frac{1}{x}\cdot(\frac{x^3}{3}))dx$\\
  $=ln(x)\cdot(\frac{x^3}{3})-\frac{1}{3}\int_{1}^{e}x^2dx$\\
  $ln(x)\cdot(\frac{x^3}{3})-\frac{1}{3}[\frac{x^3}{3}]$\\
  $ln(e)(\frac{e^3}{3})-\frac{1}{3}(\frac{e^3}{3})-(ln(1)\cdot(\frac{1^3}{3}-\frac{1}{3}\cdot(\frac{1^3}{3})))$\\
  $=\frac{e^3}{3}\cdot (1-\frac{1}{3})-0+\frac{1}{9}$\\
  $=\frac{2}{3\cdot3}e^3+\frac{1}{9}$\\
  $=\frac{2e^3+1}{9}$\\
  3. $v(t)=cos(t^2)$ \\
  i displacement=$\int(v(t))dt$\\
  $\int_{0}^{3}cos(t^2)dt=0.703$ settings in radians.\\
  ans: 0.703meter\\
  ii) total distance travelled=$\int_0^3 |cos(t^2)|dt=2.05$\\
  2.05meter\\
  
  4. \\
  5. $\int_{1}^{2}((x-2)^2+\frac{1}{x}+ sin\pi x)dx$\\
      
  


  \end{document}
