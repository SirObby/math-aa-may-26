\documentclass{article}
\title{\LaTeX Math notes}
\author{Samuel Hautamäki}
\date{16th of August 2024}
\usepackage{mathtools,amssymb,amsthm,gensymb,textcomp}
\begin{document}
  \maketitle
  
  \subsection{homework}
  HL ex.4,5,a,b,c p 116\\
  5b) $f(x)=\frac{x-5}{x-3} g(x)=\frac{-2}{x-1}+3$\\
  aer inverses if $f(g(x))=x$ and $g(f(x))=x$\\
  Lets calculate\\
  $f(g(x))=f(\frac{-2}{x-1}+3)=\frac{(\frac{-2}{x-1}+3)-5}{(\frac{-2}{x-1}+3)-3}$\\
  $=\frac{\frac{-2}{x-1}-2}{\frac{-2}{x-1}}=\frac{\frac{-2}{x-1}-\frac{2(x-1)}{x-1}}{\frac{-2}{x-1}}=\frac{-2-2x+2}{x-1}\cdot (\frac{x-1}{-2})$\\
  $=\frac{-2-2x+2}{-2}=\frac{-2x}{-2}=x$\\
  4. $f(x)=(x-1)^2$\\
  to find the inverse $y=(x-1)^2$\\
  $\sqrt{y}=x-1$\\
  $x=1+\sqrt{y}$\\
  so $f^{-1}(x)=1+\sqrt{x}.$\\
  $g (x)=2x+1$ to find the inverse\\
  $y=2x+1$ sovle for x\\
  $y-1=2x$\\
  $\frac{y-1}{2}=x$\\
  $y=\frac{x-1}{2}$\\
  so $g^{-1}(x)=\frac{x-1}{2}$\\
  Now LHS (=lefthandside)\\
  $f^{-1}\circ g^{-1} (x)=f^{-1}(g^{\frac{
    x-1
  }{2}}=1+\sqrt{\frac{x-1}{2}}$\\
  and RHS (=right hand side)\\
  $(g\circ f)^{-1}(x)=(g(f(x)))^{-1}=(g((x-1)^2))^{-1}$\\
  $=(2\cdot(x-1)^2+1)^{-1}$\\

  \section{Limit of function and sequences}
  (=funktion raja-arvo)\\
  picture has a limit when x gets closer to right side infinity then f(x) gets closer / approaches value 2.\\
  $\lim_{x\to\infty} (f(x))=2$\\
  also if x is close to -10 then function is close to $-\infty$ $\lim_{x\to-10^+}(f(x))=-\infty$\\
  and $\lim_{x\to-10^-}(f(x))=-\infty$\\
  and $\lim_{x\to0}(f(x))=\infty$\\
  To solve the limit of function:\\
  1) graphical method. (just look at it) \\
  2) numerically from table of x and y\\
  3) algebrically method by cancelling(=supistaminen murtoluvulle) the fraction and then substitute.\\
  If there is a unique value L then function is called convergent (=suppeneva) else divergent (=hajaantuva)\\
  $L=\lim_{x\to a}(f(x))$\\
  
  Example. Find $\lim_{x\to3}(\frac{x^2-9}{x-3})$
  Lets factorise by common factor. \\
  $\lim_{x\to3}(\frac{(x+3)\cdot(x-3)}{(x-3)\cdot1})$\\
  $\lim_{x\to3}(\frac{x+3}{1})=\frac{3+3}{1}=6$\\
  after cancelling try to substitute if possible.\\
  OR table method\\
  x      $y=f(x)=\frac{x^2-9}{x-3}$\\
  3.9    $\frac{(3.9^2-9)}{3.93}=6.9$\\
  2.9  $\frac{(2.9)^2-9}{2.93}=5.9$\\
  2.9999999 $\frac{2.9999999^2-9}{2.9999999-3}=6$\\
  3.001      $\frac{3.000000001^2-9}{3.00000-3}=6$\\
 it seems that limit is 6.\\
  \section{hl p 223 graphical or cancelling.}
  1) $-4$\\
  2) $6$\\
  3) $3, -1$\\
  4) $3$\\

  
  

   
\end{document}
