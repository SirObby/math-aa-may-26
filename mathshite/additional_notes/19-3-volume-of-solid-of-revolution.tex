\documentclass{article}
\title{\LaTeX Math notes}
\author{Samuel Hautamäki}
\date{th of October 2024}
\usepackage{mathtools,amssymb,amsthm,gensymb,textcomp}
\begin{document}
  \maketitle
   
  \section{Volume of solid of revolution}
  HL P 524\\
  $V= Area\cdot dx$\\
  $=\pi r^2\cdot dx$\\
  $=\int_{0}^{2}\pi\cdot(d(x))^2dx$\\
  A function $f(x)$ rotates about x-axis $360\degree (2\pi radians)$ between $a<x<b$\\
  volume $=\int_{a}^{b}(f(x))^2dx$\\
  Example 1. \\
  A line $y=3x$ rotates around x-axis $0\leq x\leq 5$. Find the volume of solid formed.\\
  $V=\pi\cdot\int_{0}^{5}(3\cdot x)^2dx=1178.1$\\
  or $\pi\cdot\int_{0}^{5}9x^2dx=\pi[3x^3]=\pi(3\cdot5^3-3\cdot0^3$\\
  $\pi\cdot 3\cdot 125=375\pi$\\
  Example 2\\
  function $f(x)=\sqrt{\frac{4-x^2}{8}}$, $-2\leq x\leq2$ rotates 360 about x-axis\\
  volume$=\pi\cdot\int_{-2}^{2}(\sqrt{\frac{4-2x^2}{8}})^2dx=\frac{4\pi}{3}$\\
  p 527 ex:8C:1a,2-4\\
  1. $f(x)=\sqrt{x}$, $1\leq x\leq 4$ rotates\\
  $volume=\pi\cdot\int_{1}^{4}\sqrt{x}^2dx=\pi \int_{1}^{4}xdx$\\
  $\pi[\frac{x^2}{2}]=\pi(\frac{4^2}{2}-\frac{1^2}{2})=\pi(\frac{16}{2}-\frac{1}{2})=\frac{15}{2}\pi$\\
  ex 2. a. \\
  $V=\pi\int_{0}^{1}(x^2+x)^2dx$\\
  $=\pi\int_{0}^{1}(x^4+2\cdot x^2\cdot x+x^2)dx$\\
  note $(a+b)^2=a^2+2ab+b^2$\\
  $=\pi\int_{0}^{1}(x^4+2\cdot x^3+x^2)dx$\\
  $=\pi[ \frac{x^5}{5} +2\cdot (\frac{x^4}{4})+\frac{x^3}{3}]$\\
  $=\pi\cdot(\frac{1^5}{5}+2\cdot (\frac{1^4}{4})+\frac{1^2}{3})-(\frac{0^5}{5}+2\cdot (\frac{0^4}{4})+(\frac{0}{3}))$\\
  $\pi\cdot(\frac{1}{5}+\frac{2}{4}+\frac{1}{3}-0)=\pi\cdot (\frac{1\cdot2\cdot3+1\cdot5\cdot3+1\cdot2\cdot5}{30})$\\
  $=\pi\frac{31}{30}$\\
  2b) $V=\int_{1}^{4}(1-\sqrt{x})^2dx$\\
  $\int_{1}^{4}(1^2-2\cdot 1\cdot \sqrt{x}+x)dx$\\
  $\int_{1}^{4}(1-2\cdot x^{1/2}+x)dx$\\
  $=\pi[ 1x -\frac{2\cdot x^{3/2}}{3/2}+\frac{x^2}{2}]$\\
  $=\pi[1x-\frac{(2\cdot x^{3/2}\cdot 2)}{3}+\frac{x^2}{2}]$\\
  $=\pi4- \frac{4}{3}\cdot 4\cdot\sqrt{4}+\frac{4^2}{2}-(1-\frac{4}{3}-1\cdot\sqrt{1}+\frac{1^2}{1})$\\
  $\pi(4-\frac{32}{3}+\frac{16}{2}-1\frac{4}{3}-\frac{1}{2})$\\
  $\pi(\frac{24-32\cdot2+16\cdot3-6+4\cdot2-1\cdot3}{6})$\\
  $=\pi\frac{7}{6}$\\
  2c) $y=\sqrt{cos(x)sin(x)}$\\
  $v=\pi\int_{0}^{\frac{\pi}{2}}(cos(x)sin(x))dx$\\
  $\pi[\frac{sin^2x}{2}]$\\
  $\pi(\frac{sin^2\frac{\pi}{2}}{2}- \frac{sin^20}{2})$\\
  $\pi(\frac{1^2}{2}-\frac{0^2}{2})=\pi\frac{1}{2}$\\
  2d) $v=\int_{0}^{\pi/4}(tan(x))^2dx$\\
  formula booklet\\
  $1+tan^2x=sec^2x$\\
  $\pi\int_{0}^{\pi/4}(sec^2x-1)dx$\\
  $\pi [tan (x) - x]$\\
  $=\pi(tan(pi/4)-pi/4 - (tan(0)-0))$\\
  $\pi(1-pi/4)$\\


  
\end{document}
