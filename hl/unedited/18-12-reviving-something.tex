\documentclass{article}
\usepackage{graphicx} % Required for inserting images
\usepackage{hyperref}
\usepackage{mathtools,amssymb,amsthm}

\title{The day's notes}
\author{Samuel Hautamäki}
%\date{October 2025}
\begin{document}
\subsection{Differentiation}
\obeylines

at min $f'(x)=0$ and $f''(x)>$concanes up
at max: $f'(x)=0$ and $f''(x)<$ concaves down
at niflexion $f''(x)=0$ concavity changes

f increases when f'(x)>0
f decreases when f'(x)<0

part 1 example question thingy idk
$f(x)=1/3x^3-x^2-3x$
at max point A: $f'(x)=0$ and $f''(x)<0$
$f'(x)=\frac{1}{3}\cdot 3x^2-2x-3=x^2-2x-3=0$
$a=1,b=-2,c=-3$
$x=\frac{2\pm\sqrt{(-2)^2-4\cdot 1\cdot (-3)}}{2\cdot 1}=\frac{2\pm\sqrt{4+12}}{2}=\frac{2\pm\sqrt{16}}{2}$
$x=\frac{2+4}{2}=3$ or $x=\frac{2-4}{2}=\frac{-2}{2}=-1$
$f''(x)=D(x^2-2x-3)=2x-2$
$f''(3)=2\cdot 3-2=4>0$ so min at $x=3$
$f''(-1)=2\cdot(-1)-2=-4<0$ so max at $x=-1$
$y=f(-1)=1/3(-1)^3-(-1)^2-3(-1)=\frac{5}{3}$
$y=f(3)=\frac{1}{3}\cdot 3^3-3^2-3\cdot3=-9$
Ans: $A=(-1,5/3)$ and $B=(3,-9)$
at inflexion $f''(x)=0$ so $2x-2=0$
$2x=2$
And $C=(1,-11/3)$

$f'(x)=slope of tangent$
$=\lim_{h\to0}(\frac{f(x+h)-f(x)}{h})$
$=\lim_{h\to0}(\frac{\frac{1}{x+h}-\frac{1}{x}}{h})=\lim_{x\to0}(\frac{\frac{1\cdot x}{x(x+h)-\frac{1(x+h)}{x(x+h)}}}{h})$
$=lim_{h\to0}(\frac{\frac{x-x-h}{x(x+h)}}{h})=\lim_{h\to0}(\frac{-h}{x(x+h}\cdot\frac{1}{h})$
$=\lim_{h\to0}(\frac{-1}{x(x+h)})=\frac{-1}{x(x+0)}=\frac{-1}{x^2}$


\end{document}
