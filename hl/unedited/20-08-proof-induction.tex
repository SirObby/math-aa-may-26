\subsect{Proof by induction}

HL P 45.

There are times direct proof is not possible.

Then we need to revert to a different proof which is called "proof by induction".

- The starting point of the process is called the \textbf{"basic step"}. (when $n=1$, we prove that $P(1)$ is true)

- Then we make an assumption. We assume that $n=k$, $P(k)$ is true for some value of k. where $k\geq 1$, $k\in\mathbb{Z}$

- Then the inductive step we must use the above assumption to show that the given mathematical statement ($P(n)$) is true. $n=k+1$

- Now the final statement completes the proof and should always be included.

Page 50 1G.

2. a) Use mathematical induction to prove the folowing statements.
$P(n)=1^2+2^2+3^2+...+n^2=\frac{1}{3}n(n+1)(n+\frac{1}{2})$

Basic step, $P(1)=1^2=\frac{1}{3}\cdot1\cdot2$ $n=1$.
$P(1)=1=1$
The statement is true for $n=1$ $\rightarrow$ $P(1)$ is true.

Assume the statement is true for $n=k$ when $k=\in\mathbb{Z}^+$

$1^2+2^2+3^3+k^2=\frac{1}{3}k(k+1)(k+\frac{1}{2})$

- Inductive step $n=k+1$.
$P(k+1)=1^2+2^2+3^2+...+k^2+(k+1)^2=\frac{1}{3}k(k+1)(k+\frac{1}{2})+(k+1)^2$

$\frac{1}{3}(n+1)[k(k+\frac{1}{2})+3(k+1)]$

$\frac{1}{3}(k+1)[N^2+\frac{1}{2}k+3k+3]$

$\frac{1}{3}(k+1)[N^2+\frac{7}{2}k+3]$

$P(k+1)=\frac{1}{3}(k+1)(k+2)(k+\frac{3}{2})$

$\frac{1}{3}(k+1)(k+1+1)(k+1+\frac{1}{2})$

Final statement.
Since $P(1)$ is true and $P(k)\rightarrow P(k+1), k\in\mathbb{Z}^+$

Then by principle of mathematical induction the statement is true for all positive integers

d) $P(n) 9^n-1$ is divisible by 8 for all $n\in\mathbb{N}$

Basic step $P(1)=9-1=8$ $8/8=1$.
$P(1) is true.$

Assumption $n=k$ $9^k-1=8mm m\in\mathbb{N}$

Inductive step $n=k+1$ $9^{k+1}-1$

$9^k\cdot 9^1-1$

from assumption subst $9^k=8m+1$.
$(8m+1)\cdot 9-1$

$9\cdot 8m+9-1$

$8\cdot (9m)+8$
$P(k+1)\rightarrow 8(9m+1)$

Final statement $P(k)\rightarrow P(k+1), k\in\mathbb{N}$

Then by the principle of mathematical induction the statement is true for all natural numbers.

b) $1-4+9-16+...+(-1)^{n+1}+n^2=(-1)^{n-1}\frac{n(n+1)}{2}$

Basic step $P(1)=1=(-1)^{1+1}\frac{1(1+1)}{2}$
$(-1)\cdot(-1)\cdot\frac{1\cdot 2}{2}=1$

- Assumption $P(k)$ is true for $n=k$ $k\in\mathbb{Z}^k$

$1-4+9-16+...+(-1)^{k+1}k^2=(-1)^{k+1}\frac{k(k+1)}{2}$

- Inductive step $P(k+1) n=k+1$

- $1-4+9-16+...+(-1)^{k+1}k^2+(-1)^{k+2}(k+1)^2$
$(-1)^{k+1}\frac{k(k-1)}{2}+(-1)^{k+2}(k+1)^2$

$(-1)^{k+1}(k+1){\frac{k}{2}+(-1)^1(k+1)}$

$(-1)^{k+1}(k+1){\frac{k}{2}-k-1}$

$(-1)^{k+1}\cdot(k+1)(\frac{-k-2}{2})$

$(-1)^{k+1}\cdot(k+1)(-1)(\frac{k+2}{2})$

$(-1)^{k+2}(k+1)(\frac{k+1+1}{2})$
