\documentclass{article}
\usepackage{graphicx} % Required for inserting images
\usepackage{hyperref}
\usepackage{mathtools,amssymb,amsthm}

\title{The day's notes}
\author{Samuel Hautamäki}
%\date{October 2025}
\begin{document}
\subsection{Finals of derivatives of polynomes}
\obeylines

$f'(x)=D(3x^2+5x+4)=3\cdot 2\cdot x^{2-1}+5\cdot 1\cdot x^0+0$
$=6x+5$
since D constant=0
$D constant\cdot x=constant$
$D x^n=n\cdot x^{n-1}$
$D x^{-1}=-1\cdot x^{-2}=\frac{-1}{x^2}$
$D \sqrt[n]{x}=D x^{1/n}=\frac{1}{n}\cdot x=(\frac{1}{n}\cdot x^{1/n-n/n})=\frac{1}{n}x^(1-n)/n$
$product D f\cdot g=f'\cdot g+f\cdot g'$
$division D(\frac{f}{g})=\frac{f'g-f\cdot g'}{g^2}$
meaning $f'=slope of tangent in a point$
$f'>0$ then f increasing 
$f<0$ decreasing
$f'(x)=0$ there is possibility to min/max/inflexion
pedanet -> derivatives SL/HL

eggcercise paper 1 style hell yeah
$y=x\cdot ln(x)^2$
min and max at f'(x)=0
product rule to differentiate 
$y'=1\cdot ln(x)^2+x\cdot 2\cdot(ln x)\cdot (\frac{1}{x})$
$=ln x\cdot (ln x+2)=0$ if 
$ln x=0$ or $ln x+2=0$
$x=1$       $ln x=-2$
$e^{-2}=x$ or $x=\frac{1}{e^2}$
point A has $x=\frac{1}{e^2}$ and $y=x\cdot (ln x)^2$=
$\frac{1}{e^2}\cdot (ln (e^{-2}))^2=\frac{1}{e^2}(-2)^2=\frac{4}{e^2}$
point B has $x=1$ $y=x\cdot ln(x)^2=1\cdot ln(1)^2=0$

2. 
$f(x)=\frac{ln x}{x}, x>0$
$f'(x)=\frac{\frac{1}{x}\cdot x-ln x\cdot 1}{x^2}$
$f'(x)=\frac{1-ln x}{x^2}$
$f'(x)=0$
$\frac{1-ln x}{x^2}=0$
$x=2.718$
$f(2.718)=\frac{ln 2.718}{2.718}=0.3678$
$(2.718,0.367)$

3
$f(x)=\frac{x+1}{x^2+1}$
$f'(x)=$


\end{document}
