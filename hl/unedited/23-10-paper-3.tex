\documentclass{article}
\usepackage{graphicx} % Required for inserting images
\usepackage{hyperref}
\usepackage{mathtools,amssymb,amsthm}

\title{The day's notes}
\author{Samuel Hautamäki}
%\date{October 2025}
\begin{document}
\subsection{Paper}
P 752
\obeylines

Any topic, 55 min with calculator.

1. 
a) sequence $u_{n+1}=au_n+b$ and $a=1$
i) substitute: $u_{n+1}=1\cdot u_n+b$ is called arithmethic sequence.
$u_2=1\cdot u_1+b$
$u_3=1\cdot u_2+b=1\cdot(1\cdot u_1+b)+b=u_1+2b$
$u_4u_3+b=u_1+3b$
ii) $u_n=u_1+(n-1)\cdot b$
iii) sum of arithmethic $s_n=\frac{n\cdot (2u_1+(n-1)\cdot b)}{2}$
b) $b=0,a\neq 1, a\neq 0$
i) $u_{n+1}=au_n$ geometric.
common ratio $=a$
ii) $u_n=u_1\cdot a^{n-1}$
iii) sum of geometric $s_n=\frac{u_1\cdot(a^n-1)}{a-1}$
c) $u_1=1$ $u_{n+1}=2u_n+1$
$u_2=2\cdot u_1+1=2\cdot 1+1=3$
$u_3=2\cdot u_2+1=2\cdot 3+1=7$
$u_4=2\cdot u_3+1=2\cdot 7+1=15$
$u_5=2\cdot u_4+1=2\cdot 15+1=31$
ii) rule general term $u_n=2^n-1$
iii) mathematical induction.
LHS $u_{n+1}=2\cdot u_n+1$ is equal to RHS $u_n=2^n-1$
1) basic step 
subst. $n=2$ lhs gives $u_2=2\cdot u_1+1=2\cdot1+1=3$ rhs gives $2^2-1=4-1=3$
so basic step P(2) is true.
2) assumption: Lets assume that $P(k)$ is true.
that is LHS $u_k+1=2\cdot u_k+1$
equal to rhs $u_k=2^k-1$
3) induction step substitute $n=k+1$
LHS $u_n+1=u_{k+1}+1=2\cdot(u_{k+1})+1=use above$
$=2\cdot (2\cdot u_k+1)+1=2\cdot (2(2^k-1)+1)+1$
$=2^{k+1+1}-2\cdot2\cdot1+2\cdot 1+1$
$2^{(k+1)+1}-4+2+1=2^n+1-1=rhs$
so $P(k+1)$ is true
4) by the principle of math induction $P(n)$ is true for all integer value.

d. $u_{n+1}=a\cdot u_n+b$
has relation $u_{n+1}+c=a\cdot(u_n+c), c=?$
subst. $a\cdot u_n+b+c=a\cdot(u_n+c)$
$a\cdot u_n+b+c=a\cdot u_n+ a\cdot c$
$b+c=a\cdot c$
$c-ac=-b$
$c\cdot (1-a)=-bc$
$c=\frac{-b}{1-a}$ or 
$c=\frac{b}{a-q}$
ii) $v_n=u_n+c=(a\cdot (u_{n-1})+b)+c$
$v_n=a\cdot u_{n-1}+b+\frac{b}{a-1}$
recurrence relation
iii) $v=u_1+c=1+\frac{b}{a-1}$
$v_2=u_2+c=$


\end{document}
