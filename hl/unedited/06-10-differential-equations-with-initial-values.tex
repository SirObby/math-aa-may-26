\documentclass{article}
\usepackage{graphicx} % Required for inserting images
\usepackage{hyperref}
\usepackage{graphicx} % Required for inserting images
\usepackage{mathtools,amssymb,amsthm}

\title{The day's notes}
\author{Samuel Hautamäki}
%\date{October 2025}
\begin{document}
\subsection{Differential equations with initial values}
HL p 536
\obeylines
1) ordinary

2) homogenious

3) integrating factor method to solve y from the equation.

constant of integration c can be solved w ith initial value example by substitution of $y(0)=2$

Example. Solve $3x\cdot(\frac{dy}{dx})-2=2y^2$ and $y(e)=0$

Let's try to separate

$3x\cdot \frac{dy}{dx}=2+2y^2$ 
$3x dy=(2+2y^2)dx$
$\frac{3x}{2+2y^2}dy=dx$
$\frac{1}{2+2y^2}dy=\frac{dx}{3x}$
$\int\frac{1}{2+2y^2}dy=\int\frac{1}{3x}dx$
$\frac{1}{2}\int\frac{1}{2+y^2}dy=\frac{1}{3}\int\frac{1}{x}dx$
$\frac{1}{2}arctan(y)=\frac{1}{3}ln(x)+c$
$arctan(y)=\frac{2}{3}ln(x)+c$
$y=tan(\frac{2}{3}ln(x)+c)$ and initial value $y(e)=0$
substitution $arctan(0)=\frac{2}{3}ln(e)+c$
$0=\frac{2}{3}log_e(e^1)+c$
$0=\frac{2}{3}1+c$
$c=\frac{-2}{3}$
Ans: $y=tan(\frac{2}{3}ln(x)-\frac{2}{3})$

Example page 537.
$\frac{dT}{dt}=-k(T-T_0)$
$\frac{dT}{dt}=-k(T-22)$
$\frac{dT}{T-22}=-kdt$
$\int\frac{dT}{T-22}=\int-kdr$
$ln(T-22)=-kt+c$
$e^{-kt+c}=T-22$
$Ae^{-kt}=T-22$
When t=6,T=30 and when t=9, T=26
$Ae^{-6k}=30-22=8, Ae^{-9k}=26-22=4$
$e^{3k}=\frac{8}{4}\Rightarrow A=8e^{2ln2}=8\cdot 4=32$
SSo $T=32e^{-\frac{tln2}{3}}+22$
Person died when $T=37$
$37=32e^{-\frac{tln2}{3}}+22$
$e^{-\frac{tln2}{3}}=\frac{15}{32}$
$e^{\frac{tln2}{3}}=\frac{32}{15}$
$t=\frac{3}{ln2}(ln(\frac{32}{15}))=3.28$
The person died approx 3.28 hours after midnight, that is 03:17am
p 539 do ex 8F:1,2,...11+p546 ex:2
1. direct proportion to rate of change
$\frac{dT}{dt}=k\cdot (T-T_0)$
where $T_0$ is constant temperature of laboratory
b) time 10 min to cool down $100\rightarrow 80$
and further 10 min $80\rightarrow 65$
and time=0 then T=100
Lets first solve $\frac{dT}{dt}=k\cdot(T-T_0)$
$\frac{dT}{T-T_0}=k\cdot dt$ integrate
$\int\frac{1}{T-T_0}dT=\int k dt$
$ln(T-T_0)=k\cdot t+c$
$e^{k\cdot t+c}=T-T_0$ and $e^{c}=A$
$a\cdot e^{kt}=T-T_0$ substitute
$\begin{cases}a\cdot e^{k\cdot 0}=100-T_0
\\ a\cdot e^{k\cdot 10}=80-T_0
\\ a\cdot e^{k\cdot 20}=65-T_0\end{cases}$
subst $e^{k\cdot 10}=x, e^{k\cdot 20}=y, z=T_0$
$\begin{cases}a1=100-z
\\ a\cdot x=80-z
\\ a\cdot y=65-z\end{cases}$
1st equation.
$a\cdot e^{k\cdot 0}=100-T_0$
$a=100-T_0, T_0=100-a$
when $t=10, T=80$ gives 
$80=T_0+ae^{10k}$
$80=100-a+aek^{10k}$
$-20=a(-1+e^{10k})$
$a=\frac{-20}{-1+e^{10k}}$
$a=\frac{20}{1-e^{10k}}$
3th equation $a\cdot e^{k\cdot20}=65-T_0$
$\frac{20}{1-e^{10k}}\cdot e^{20k}=65-(100-\frac{20}{1-e^{10k}}$
if $x=e^{10\cdot k}$ then $e^{20k}=e^{10k\cdot 2}=(e^{10k})^2=x^2$
$nSolve(\frac{20}{1-x}\cdot x^2=65-(100-\frac{20}{1-x},x)$
$x=0.75=e^{10k}$
$e^{10k}=\frac{3}{4}$
$10k=ln(\frac{3}{4})$
$k=\frac{1}{10}ln(\frac{3}{4})$.
And now $T_0=100-a=100-(\frac{20}{1-x})=$
$100-\frac{20}{1-\frac{3}{4}}=20$
Ans: temp of lab is 20C
c) anoter 10 min means time $t=30$
$T(30)=?     a\cdot e^{kt}=T-T_0$
$\frac{20}{100-3/4}\cdot^{30\cdot\frac{1}{10}ln(3/4)}+20$
$=80\cdot e^{3\cdot ln(3/4)}+20=53.75$
Ans: 53.8C

Note! rate of change $=\frac{dT}{dt}=\frac{dP}{dt}=$
or kinematics
$a(t)=\frac{dvelocity}{d time}$ and $v(t)=\frac{ds}{dt}$

\end{document}
