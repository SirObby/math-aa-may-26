\documentclass{article}
\usepackage{graphicx} % Required for inserting images
\   usepackage{hyperref}
\usepackage{mathtools,amssymb,amsthm}

\title{The day's notes}
\author{Samuel Hautamäki}
%\date{October 2025}
\begin{document}
\subsection{Integration and area bounded by curve}
\obeylines
for a positive function $f(x)$ and $a<x<b$
area bounded by curve $\int_{a}^{b}f(x) dx$
and for a negative function
$area=-\int_{a}^{b}f(x) dx$
bounded are may have upper and lwoer limit from zeroes of function
$f(x)=0$
$x=a$ or $x=b$
Example find the area bounded by curve $f(x)=-(x+2)(x-3).$
curve is parabola, downwards
zeroes: $x+2=0$ $x-3=0$
$x=-2$  $x=3$
paper 1: Area=$\int_{-2}^{3}(-(x+2)(x-3))dx$
$=\int_{-2}^{3}(-x^2+3x-2x+6)dx=\int_{-2}^{3}(-x^2+x+6)dx$
$=-\frac{1}{3}x^3+\frac{1}{2}x^2+6x$
$=\frac{-1}{3}\cdot 3^3+\frac{1}{2}\cdot 3^2+6\cdot 3-(\frac{-1}{3}(-2)^3+\frac{1}{2}\cdot(-2)^2+6\cdot-2)$

integration:3 exam-line questions.
1. $f(x)=ln(\frac{2}{e^x+1})$
a) Domain is all real numbers.
Range $\lim_{x\to-\infty}(ln(\frac{2}{e^x+1}))$
$ln(2)=0.069$
b) 
Area=$4=-\int_{0}^{k}ln(\frac{2}{e^x+1})dx=$
$=-2\int_{0}^{k}(ln(1)-ln(e^x+1))dx$
nSolve.
Ans: $k\approx3.32$

1. The graph represents the funcntion
$f: x\mapsto p cos(x), p\in\mathbb{N}$
means that $f(x)=p\cdot cos(X)$
where p is any natural number $(=0,1,2,3,...)$
a) ans: p=1
b) area=?=$\int_{0}^{\frac{\pi}{2}}(1\cdot cos(x))dx$
$[sin(x)]$
$sin(\pi/2)-sin(0)$
$=1-0=1$



\end{document}
