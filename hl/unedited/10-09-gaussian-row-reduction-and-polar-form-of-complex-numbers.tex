\subsection{Gaussian method}
p 207

Example solve $\begin{cases}
x+2y+z=6[1] \\ 
x-y-z=-1[2] \\
3x-y-2z=2[3]
\end{cases}$ by Gaussian method.
Let's eliminate x by $R_2->R_2-R_1$ and $R_3->R_3-3R_1$
$\begin{cases}
1x+2y+1z=6[1] \\ 
0x-3y-2z=-7[2] \\
0-7y-5z=-16[3]
\end{cases}$
Now eliminate $y$ from [3]->3[3]-7[2]
$\begin{cases}
1x+2y+1z=6[1] \\ 
0x-3y-2z=-7[2] \\
0+0+z=-1[3]
\end{cases}$
so $z=-1$ and subst. $[2]$ $3y-2\cdot(-1)=-7$
$-3y=-9$
$y=3$
and subst [1] $1x+2y+1z=6$
$x+2\cdot 3-1=6$
$x=1$
Ans: $x=1,y=3,z=-1$
Gdc linsolve.

\subsection{Polar form of complex numbers}
p 650

Complex number has forms:
1) $Z=x+yi$ where x is real part of z and y is factor of imaginary part.
Called cartesian, Argand plane form (or Rectangular on gdc).
2) $z=cos(\alpha)+sin(\alpha)\cdot i$
where $r=|z|=\sqrt{x^2+y^2}$
and $\alpha=tan^{-1}(\frac{y}{x})$ $\pm\pi$in degrees or radians.
check the sketch.
real part of $z$ is $x=cos(\alpha)\cdot r$
imaginary part of $z$ is $y=sin(\alpha)\cdot r$
3) $z=r\cdot cis(\theta)=r\cdot e^{i\theta}$
where $r\cdot cis(\theta)=r\cdot (cos(\theta)+i\cdot sin(\theta)$
$r\cdot e^{i\theta}$ is called Euler's form.

Example. a) Convert into Polar form 
$z=3+2i$
$r=\sqrt{x^2+y^2}=\sqrt{3^2+2^2}=\sqrt{13}$
angle $\theta=tan^{-1}(\frac{y}{x})=tan^{-1}(\frac{2}{3})=0.588$ radians
and angle is in the first quadrant.
and $y=2>0$
Ans: polar form $z=rcis(\theta)=\sqrt{13}\cdot cis(0.588)$
b) Convert into Cartesian form.
$z=5cis(53,1\degree)$
Cartesian $z=x+yi$
real part $x=cos(53.1\degree)\cdot 5=3.0021=3$ (3.sf)
factors of imaginary part $y=sin(angle)\cdot r=sin(53.1\degree)\cdot 5=3.998=4$ (3.sf)
Ans: $z=3+4i$
c) Convert $z=3+4i$ into Euler's form
$z=3+4i$ has $r=5$ and angle $=53.1\degree$
Euler's $z=r\cdot e^{i\cdot\theta}$
$=5\cdot e^{i\cdot 53.1\degree}$
or radian angle $\degree=\frac{53.1\degree\cdot \pi}{180\degree}=0.927$radians
$Ans: z=5\cdot e^{0.927i}$
ex p 655 ex1ab,2ab,3ab,4ab
ex 1. 
a) $|z|=r=1$ and arg $z=\theta=\pi$
$z=-1+0i$
b) $|x|=r=2$ and arg $z=\theta=\frac{\pi}{2}$
$z=0+2i$
ex 2.
a) $z=2+2i$
$r=\sqrt{2^2+2^2}=\sqrt{8}$
$angle theta=\tan^{-1}(\frac{2}{2})=0.785$
$z=\sqrt{8} cis(angle)=\2\cdot\sqrt{2}cis(0.785)$

b) 
