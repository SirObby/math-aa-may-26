\subsection{Applications of series}

hl p 28.

Read carefully and try to 'sketch'

Arithmetic $S_n=u_1+u_2+...+u_n=\frac{u_1+u_n}{2}\cdot n$

Geometric $S_n=\frac{u_1\cdot(1-r^n)}{1-r},r\neq1$
and $S_\infty=\frac{u_1}{1-r}$ for $-1<r<1$

Applications into bank interest, population growth/decay etc.

See HL p 29, example 26.

Simple interest $A_n=P(1+r\cdot n)$
where $n=number of years$ 
$r=interest factor$ ($13%=0.13$)
$P=capital$
Compound interest $A=P(r)^n$ %$A_n=P\cdot(1+\frac{r}{n})^{n\cdot t}$
%(n refers to number of compounding periods in one year, quarterly=$n=4$, t=years)

Example. Sebastian took out a loan for a new car that cost $35000$. The bank offered him 2.5\% per annum simple interest for five years. Calculate the total value. 

$A=P(1+nr)=35000(1+5(0.025))$
$A=35000(1.125)$
$A=39375$

HL p 31 ex:1-10.
1. b) $d=+5$
year 2010: $u_1=220$
year 2017: 7 years twice a year: $u_n=u_1+d(n-1)=220+10(7-1)=290$

b) $\frac{220+290}{2}\cdot 8=2040$

c) $d=-20$
$600/2=300$.
$u_n=u_1+d(n-1)=220+10(n-1)=300$
$n=9$

2. $49650=p\cdot(1.015)^11$
Ans: $42150$

3. a) $2^4=16$ Book says 30, but this cannot be the case. $2^1=2=u_1, 2^2=4,u_2$. As such $u_n=2^n$, but book says $u_4=30$?

b) $2^{n+1}>1,000,000$
$n=19$ generations back.

