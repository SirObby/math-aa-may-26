\subsection{Powers and roots of complex number}
HL p 664.

De Moive's theorem $z^n=(r\cdot cis(\theta))^n=r^n\cdot cis(n\theta)$
formula booklet.

Roots: solve the equation
$z^n=r\cdot cis(\alpha)$
$z=\sqrt[n]{r}cis(\frac{\alpha+2\pi\cdot k}{n}) for k=0,1,2...(n-1)$

Example. Solve $z^3=27i$
Change RHS to Polar Form.
$r=|z|=\sqrt{a^2+b^2}=\sqrt{27^2}=27$
$\theta=tan^{-1}(\frac{27}{0})$ so $27i$ is on y(imaginary)-axis that has $90\degree=\pi/2$.

Now we have $z^3=27\cdot cis(\frac{\pi}{2})$
$z=\sqrt[3]{27} cis(\frac{\frac{\pi}{2}+2\pi\cdot k}{3}) where k=0,1,2$
$z_0=\sqrt[3]{27} cis(\frac{\frac{\pi}{2}+2\pi\cdot 0}{3})=3\cdot cis(\frac{\pi}{6})$
$z_1=\sqrt[3]{27} cis(\frac{\frac{\pi}{2}+2\pi\cdot 1}{3})=3\cdot cis(\frac{\pi}{6}+\frac{2\pi\cdot 2}{3\cdot 2})=3cis(\frac{\5pi}{6})$
$z_2=\sqrt[3]{27} cis(\frac{\frac{\pi}{2}+2\pi\cdot k}{3})=3\cdot cis(\frac{\pi}{6}+\frac{4\pi\cdot 2}{3\cdot2})$
$=3cis(\frac{9\pi}{6})$
Ans: Argand diagram shows that roots
1) Has same radius
2) They divide circle in $(n=3)$ arcs/sectors having angle in between $\frac{2\pi}{n}=\frac{2\pi}{3}$
c) Give the roots in Cartesian form.
$z_0=3\cos(\frac{\pi}{6})+3i\cdot sin(\frac{\pi}{6})=3\cdot\frac{3\pi}{2}+3\cdot\frac{1}{2}i$
$z_1=3\cdot cos(\frac{5\pi}{6})+3i\cdot sin(\frac{5\pi}{6})=3\cdot(-\frac{\sqrt{3}}{2}+3\cdot \frac{+1}{2}\cdot i$
$z_2=3\cdot cos(\frac{9\pi}{6})+3i\cdot sin(\frac{9\pi}{6})=3\cdot(0)+3\cdot (\frac{-1}{2})\cdot i=-3i$
p 667 ex1-2 a only.
p 667 ex1.a) find $(1+i)^3\cdot (2e^{1\cdot\frac{2\pi}{3}})^2$
De Moivre's theorem.
$=(\sqrt{1^2+i^2}\cdot cis(\frac{\pi}{4}))^3\cdot(2\cdot cis(\frac{2\pi}{3}))^2$
$=(\sqrt{2})^3\cdot cis(3\cdot\frac{\pi}{4})\cdot 2^2cis(\frac{2\pi}{3}\cdot 2)$
Let's simplify product.
$=4\cdot(\sqrt{2})^2\cdot(\sqrt{2})^1\cdot cis(\frac{3\pi}{4}+\frac{4\pi}{3})$
$=4\cdot2\cdot\sqrt{2}cis(\frac{3\pi\cdot 3}{4\cdot 3}+\frac{4\pi\cdot 4}{3\cdot 4})$
$8\cdot\sqrt{2}cis(\frac{25\pi}{12})=8\cdot\sqrt{2}cis(\frac{5\pi}{4})$
p 671 ex 1-4 a) only.
