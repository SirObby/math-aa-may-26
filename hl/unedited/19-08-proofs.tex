\subsection{Proofs}

HL p 33

NOTE:If exam says "proof something" include conconcluding statement "QED" at the end when proven, to get mark.


Example: Students dont know if $(a+b)^2-(a-b)^2$ is odd or even or neither. Can you help?

Expand.

$(a+b)^2-(a-b)^2=a^2+2ab+b^2-(a^2-2ab+b^2)$

$=a^2+2ab+b^2-a^2+2ab-b^2$
$=4ab=2\cdot(2ab)$

So even.

This is example of direct proof(=suora todistus).

Example 2.
Show that if product of two integers is even, then atleast one of integers is even.
proof by contraduction, antithesis. 

$A\rightarrow \leftarrow\rightarrow not B\rightarrow notA$

A=product is even.
not A=product is not even, so it is odd.
B=at least one of integers is even.
not B=none of the integers are even, so both are odd.

proof: not B lets assume both integers are odd.
a=$2\cdot x+1$
b=$2\cdot y+1$
Now the product $a\cdot b=(2x+1)\cdot(2y+1)$
$=4xy+2x+2y+1$
$=2\cdot(2x+x+y)+1$
is odd that means not A.
and the statement is proven.
since $not B\rightarrow not A$ means that $A\rightarrow B$

Note irrational number (=means that you can not write it as fraction)so use method that assume number is written as fraction and it leads into a contrdiction.

Example 28 p 37.
$S_n=\frac{u_1+u_n}{2}\cdot n=\frac{1+(2n-1)}{2}\cdot n=n\cdot n=n^2$

HL p 40 1-2 and p44 ex1-2
1. $(a+b)^2+(a-b)^2=2(a^2+b^2)$
$a^2+2ab+b^2+a^2-2ab+b^2=2(a^2+b^2)$
$a^2+b^2+a^2+b^2=2(a^2+b^2)$
$2\cdot(a^2+b^2)=2(a^2+b^2)$
QED. (NOTE:If exam says "proof something" include conconcluding statement "QED" at the end when proven, to get mark.)

2. 
$a=2\cdot x+1$
$b=2\cdot y+1$
$ab=(2\cdot x+1)\cdot(2\cdot y+1)=4xy+2y+2x+1$
$+1=odd$
QED

p 44
1. If $n^2$ is odd then n is odd.
counter statement: if n is not odd, then $n^2$ is not odd.
Let's assume notB means$n is not odd, n=2x$
now $n^2=(2x)^2=4x^2=2\cdot(2x^2)+0$ is even. not odd.

original statement is true, QED.
