\subsection{Operations in polar form}
p 656

$z=r cis \theta$
$-r=r cis(\theta + \pi)$
$z^*=r cis(-\theta)=rcis(2\pi-\theta)$
$-z^*=r cis(\pi-\theta)$
real factor $t z=\begin{cases}t\cdot r cis(\theta),>0
\\ 0, t=0\\
|t|\cdot r cis(\pi+\theta)\end{cases}$

Product $z_1z_2=r_1\cdot r_2 cis(\theta_1+\theta_2)$
Division $\frac{z_1}{z_2}=\frac{r_1}{r_2}cis(\theta_1-\theta_2)$
reciprocal $\frac{1}{z}=\frac{1}{r}\cdot cis(-\theta)$ since ($z^0=1$)

and same in Euler's form.

Example For $z_1=\sqrt{2}cis(\frac{\pi}{4})$ and $z_2=2cis(\frac{\pi}{3})$
a) Find the product
$z_1\cdot z_2=2\sqrt{2}cis(\frac{\pi}{4}+\frac{\pi}{3})$
$=2\sqrt{2}cis(\frac{3\pi+4\pi}{4\cdot3})$
$=2\sqrt{2}cis(\frac{7\pi}{12})$
b) Find the division.
$\frac{z_1}{z_2}=\frac{\sqrt{2}}{2}cis(\frac{\pi}{4}-\frac{\pi}{3})=2\sqrt{2} cis(\frac{\pi}{12})$
$2\sqrt{2} cis(2\pi-\frac{\pi}{12})=2\sqrt{2}cis(\frac{23\pi}{12})$

Note calculator settings in radians and compelx numbers rectangular or polar form!

p 659 ex 1a,b,2 + p663 ex1a,b,2a,b,3a,b,
1.a) $2\cdot e^{1\pi/3}\cdot 4\cdot e^{1\pi/4}$
$=8\cdot e^{\frac{7i\pi}{12}}$
paper 1 style $2\cdot 4 e^{1(\pi/3+\pi/4)}=8\cdot e^{i(7\pi/12)}$
$8e^{i\cdot\frac{7\pi}{12}}$
b) $5\cdot cis(90\degree)\cdot 6cis(45\degree)$
$=5\cdot 6cis(90\degree+45\degree)=30cis(135\degree)$
2. a) $z_1=cis(\frac{3\pi}{4})$
has $r=1$
Cartesian form=$x+iy=?$
$\sqrt{x^2+y^2}=1$ and $\frac{3\pi}{4}=tan^{-1}(\frac{y}{x})$ and 2. quadrant
memory triangle $45\degree=\frac{\pi}{4}$
so $y=\frac{1}{\sqrt{2}}=\frac{\2}{2}$ and $x=-\frac{1}{\sqrt{2}}=-\frac{\sqrt{2}}{2}$
Ans $z_1=-\frac{\sqrt{2}}{2}+\frac{\sqrt{2}}{2}i$
b) $z_2=-1/2+\sqrt{3}/2i$
polar form $r cis(\theta)$
$r=\sqrt{(\frac{-1}{2})+(\sqrt{3})^2}=\sqrt{\frac{1}{4}+3}=\sqrt{\frac{13}{4}}=\frac{\sqrt{13}}{2}$
arg$\theta=tan^{-1}(\frac{\sqrt{3}/2}{-1/2})=tan^{-1}(\sqrt{3})$ and the quadrant 2 + memory triangle $60\degree=\pi/3$
but in 2. quadrant it means $\pi-\pi/3=\frac{2\pi}{3}$
Ans: Polar form of $z_2=\frac{\sqrt{13}}{2}cis(\frac{2\pi}{3})$
c) product in both forms:
cartesian.
$(-\frac{\sqrt{2}}{2}+\frac{\sqrt{2}}{2}i)\cdot(\frac{-1}{2}+\frac{\sqrt{3}}{2}i)$
$=\frac{\sqrt{2}}{4}-\frac{\sqrt{6}}{4}i-\frac{\sqrt{2}}{4}i+\frac{\sqrt{6}}{4}i^2$
$\frac{\sqrt{2}}{4}-\frac{\sqrt{6}-\sqrt{2}}{4}i-\frac{\sqrt{6}}{4}$
$\frac{\sqrt{2}-\sqrt{6}}{4}\cdot(1+i)$ or calculator.
polar form product.
$cis(3\pi/4)\cdot(\frac{\sqrt{13}}{2}cis(\frac{2\pi}{3}))$
$1\cdot\frac{\sqrt{13}}{2}cis(\frac{3\pi}{4}+\frac{2\pi}{3})$
$=\frac{\sqrt{13}}{2}cis(\frac{17\pi}{12})$
d) Hence $sin(\frac{17\pi}{12})=imaginary of above=-\frac{\sqrt{6}-\sqrt{2}}{4}$
and $cos(\frac{17\pi}{12})$=real part of above$=\frac{\sqrt{2}-\sqrt{6}}{4}$
e) $tan(\frac{17\pi}{12})=\frac{sin()}{cos()} above=\frac{-\frac{\sqrt{6}-\sqrt{2}}{4}}{\frac{\sqrt{2}-\sqrt{6}}{4}}$
$=1$
