\subsection{Polynomial equations and inequalities}
HL p 175.

$f(x)=a_n\cdot x^n+a_{n-1}\cdot x^{n-1}+...+a_2x^2+a_1x+a_0$
a constant function $f(x)=C$
a linear function $f(x)=m\cdot x+c$
where slope=gradient=$\frac{change of y}{change of x}=f'(x)$ and passing point (0,c)
quafratic function $f(x)=a_2x^2+a_1x+a_0,a_2\neq 0$
cubic function $f(x)=a_3x^3+2_ax^2+a_1x+a_0,a_3\neq 0$
equations $f(x)=g(x)$
two functions are equal if and only if same degree and all corresponding factrs are equal.
degree($\lambda\cdot f(x)+\mu \cdot g(x))$$\leq$max$(degree f or g)$
product has the grid method / 'box method'.
Example. 
Given $f(x)=3x^3+2x^2-x+7$ and $g(x)=-3x^3+4x^2-3x+2$ Find product by box method.
$\begin{table}[]
\begin{tabular}{lllll}
      & 3x^3  & 2x^2  & -x    & +7     \\
-3x^3 & -9x^6 & -6x^5 & +3x^4 & -21x^3 \\
4x^2  & 12x^5 & 8x^4  & -4x^3 & 28x^2  \\
-3x^2 & -9x^4 & -6x^3 & +3x^2 & -21x   \\
2     & 6x^3  & 4x^2  & -2x   & x14   
\end{tabular}
\end{table}$
$f(x)\cdot g(x)=-9x+6x^5+2x^4-25x^3+35x^2-23x+14$
Division of polynomials use synthetic division, called Horner's algorithm for 1st degree divisions or long divisin for all degrees.
Example.
a) Use synthetic division $\frac{3x^3-2x^2+5x-1}{x-2}$

zeroe of divisor $x-2=0$
$x=2$
$\begin{table}[]
\begin{tabular}{lllll}
%      & 3 & -2 & -5 & -1 \\
+      &   & 6  & 8  & 6  \\
2\cdot & 3 & 4  & 3  & 5 
\end{tabular}
\end{table}$
TODO: Recap Horner's

Ans: $\frac{3x^3-2x^2-5x-1}{x-2}=3x^2+4x+3+\frac{5}{x-2}$

Example 2. Use long division $\frac{3x^3-2x^2-5x-1}{x-2}$
TODO: Recap long division.
$=3x^2+4x+3+\frac{5}{x-2}$

Polynomial remainder theorem.
when $f(x)$ is divided by $x-p$ then the remainder is $f(p)=f(zeroe of divisor)$
$\frac{g(x)}{ax+b}$ has remainder $=g(\frac{-b}{a})$
Facor theorem.
$f(x)$ has a factor $ax+b$ if $f(\frac{-b}{a})=0$ (remainder $=0$)

hl p 181 ex 1a. p 184ex 1a,2,3,4...

1.a. $3x^3+2x^2-3x+1$
$\begin{table}[]
\begin{tabular}{lllll}
%       & 1 & 2 & -3 & 1  \\
+       &   & 2 & 8  & 10 \\
\cdot 2 & 1 & 4 & 5  & 11
\end{tabular}
\end{table}$
$=\frac{3x^3+2x^2-3x+1}{x-2}=1x^2+4x+5$
Quotient $=1x^2+4x+5$
Remainder $=11$
p 184 ex1.
a. $f(x)=4x^3-5x^2+13x-2$ and $g(x)=4x-1$
$\frac{4x^3-5x^2+13x-2}{4x-1}$
$4x-1=0$
$4x=+1$
$x=\frac{1}{4}$
$\begin{table}[]
\begin{tabular}{lllll}
%                & 4 & -5 & 13 & -2 \\
+                &   & 1  & -1 & 3  \\
\frac{1}{4}\cdot & 4 & -4 & 12 & 1 
\end{tabular}
\end{table}$
Ans: Quotient $=4x^2-4x+12$
Remainder $=1$
2. $f(x)=?$ but divisor=$g(x)=2x^2-3x+1$
gives quotient $q(x)=x^2+2$ and remainder $r(x)=x-3$
$\frac{f(x)}{g(x)}=q(x)+\frac{r(x)}{g(x)}$
$f(x)=q(x)\cdot g(x)+r(x)$
$=(x^2+2)\cdot(2x^2-3x+1)+(x-3)$
$\begin{table}[]
\begin{tabular}{llll}
*   & 2x^2 & -3x   & 1   \\
x^2 & 2x^4 & -3x^3 & x^2 \\
2   & 4x^2 & -6x   & +2 
\end{tabular}
\end{table}$
$=2x^4-3x^3+5x^2-6x+2  +x-3$
Ans: $f(x)=2x^4-3x^3+5x^2-5x-1$
ex. 3.
divisible means that remainder=0 factor theorem $f(zeroe of divisor)=0$
divisor $x+2=0, x=-2$
so $f(x)=6x^5+17x^4-20x^3-35x+44x+a$
subst. $f(-2)=6(-2)^5+17(-2)^4-20(-2)^3-35(-2)+44(-2)+a$
$=6\cdot (-32)+17\cdot 16-20\cdot (-8)-35\$
t 4+44\cdot -2+a$
$12+a and =0$
$12+a=0$
$a=-12$
Ans:$a=-12$


